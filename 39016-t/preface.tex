\begin{flushright}
\textbf{PRÉFACE}
\end{flushright}

Armando Palacio Valdés est un des romanciers les plus connus de
l'Espagne. Ses œuvres ont été traduites dans la plupart des langues
européennes, et l'une d'elles, Maximina, a eu le rare bonheur d'être
tirée aux États-Unis à deux cent mille exemplaires. Après l'Amérique du
Nord, c'est en Angleterre que Palacio Valdés compte le plus
d'admirateurs. On s'y sert d'un de ses romans pour enseigner l'espagnol
dans les écoles. C'est pourquoi quelques-uns de ses compatriotes
l'accusèrent, quand il commença de publier ses sentiments aliadophiles,
de ne faire que rendre aux Alliés ce qu'il leur devait de gloire et
d'argent. Il suffira de parcourir ce livre-ci pour voir combien cette
accusation est peu fondée.

En France, plusieurs ouvrages de Palacio Valdés ont paru en feuilletons
dans nos grands quotidiens: le Capitaine Ribot, au «Gaulois», la
Sœur Saint-Sulpice, au «Matin»; la Famille Bellinchon, au «Temps»;
des extraits des Papiers du docteur Angélique, au «Journal des
Débats». On verra tout à l'heure qu'il s'en faut beaucoup que nous ayons
tout traduit du grand romancier. Il y a dans son œuvre plusieurs
romans dont il est regrettable que nous n'ayons pas d'édition
française.

\horizontalLine

Armando Palacio Valdés est né en 1854, à Entralgo, petit village des
montagnes asturiennes. Il y demeura très peu de temps, ses parents ayant
dû trans\-férer leur résidence à Avilès, une des petites villes maritimes
de la même région; mais il revint chaque année avec eux passer les mois
d'été à Entralgo. Il eut une enfance heureuse, remplie tour à tour de
jeux marins et rustiques. Les souvenirs de cette période de sa vie et de
ces lieux ont inspiré à Palacio Valdés l'Idylle d'un malade et le
Village perdu, romans de mœurs asturiennes, dont le second est
peut-être l'un des plus originaux qu'il ait écrits.

A Oviedo, capitale des Asturies, où il alla faire ses études, le jeune
Valdés se lia d'étroite amitié avec Leopoldo Alas, son condisciple, qui
devait devenir sous le pseudonyme de «Clarin» l'un des meilleurs
critiques littéraires espagnols des dernières années du siècle passé.

Son «bachillerato» terminé, Palacio Valdés s'en fut à Madrid pour faire
son droit. Cette étude le passionna. Pour s'y livrer avec plus de profit
et plus d'application, il se fit recevoir de l'Ateneo, sorte de cercle
qui comprend à Madrid tous les jeunes hommes aimant la science, les
arts ou la littérature, et dont la bibliothèque est très riche. Palacio
Valdés y dévora les traités de philosophie, d'histoire et surtout
d'économie politique. A ce moment-là, son désir le plus vif était d'être
un savant professeur. Il fut bientôt élu secrétaire de la section des
Sciences morales et politiques de l'Ateneo.

\horizontalLine

Cependant Palacio Valdés avait achevé son droit. Il commença d'écrire
et, chose curieuse chez un homme qui devait être un si abondant et si
gracieux conteur, c'est par des articles de philosophie religieuse qu'il
débuta dans les lettres. Ces articles furent remarqués. Ils valurent à
leur signataire d'être nommé rédacteur en chef de la Revista Europea,
la revue scientifique la plus importante alors en Espagne. Palacio
Valdés n'avait que vingt-deux ans.

Voulant donner plus d'attraits à sa revue, le nouveau directeur eut
l'idée d'y publier des portraits littéraires humouristiques des
principaux orateurs, romanciers, poètes et savants espagnols. Il prit à
tracer ces portraits le goût d'écrire et, poussé d'ailleurs à le suivre
par le succès de ses premiers écrits, il entreprit un roman. Commencé à
Madrid, Monsieur Octave fut terminé à Entralgo. Il parut dans les
derniers mois de 1880.

C'est avec Marthe et Marie, trois ans plus tard, que Palacio Valdés
atteignit le grand public. Le grand romancier, qui est très modeste, dit
qu'il doit le retentissant succès de ce livre au dessinateur qui
l'illustra et à l'éditeur qui le mit en vente à un prix modique. En tout
cas, Palacio Valdés était en pleine fortune: le directeur de la Revista
Europea était heureux, le romancier l'était aussi, l'homme allait
l'être; il se maria. L'Idylle d'un malade est de cette époque. Il fut
bientôt suivi de José et d'un recueil de contes intitulé
Eaux-fortes, qui consacrèrent définitivement la réputation de
l'auteur.

Ainsi tout souriait à Palacio Valdés. Il terminait Riverita, histoire
romanesque de sa propre vie, quand il perdit sa femme. Maximina, qui
parut bientôt après, est composé en grande partie en son souvenir.
Riverita et Maximina se font suite: c'est lui et elle.

Avec le Quatrième pouvoir (1888), Palacio Valdés cesse de se conter
lui-même. C'est le récit des luttes politiques dans un petit pays; mais
ici encore l'action se passe dans un milieu auquel le romancier est
étroitement attaché; la ville de Sarrio, de ce roman, n'est autre que
Gijôn, la seconde grande ville des Asturies.

Cette même année, Palacio Valdés fit un voyage en Andalousie. Il en
rapporta la Sœur Saint-Sulpice (1889), roman de mœurs andalouses
d'une exquise gaieté, qui répandit son nom dans le monde entier.

Puis ce fut l'Écume, satire de l'aristocratie espagnole, la seule de
toutes ses œuvres où Palacio Valdés, abandonnant son naturel
idéaliste, ait sacrifié aux théories littéraires alors dans toute leur
force, celles de l'école naturaliste.

Jusqu'alors il avait donné chaque année un roman. Dans la suite il mit
moins de régularité dans sa production. La Foi, le Chevalier, l'Origine
de la pensée, la Joie du capitaine Ribot, les «Majos» de Cadix, le
Village perdu, Tristan ou le Pessimisme parurent ainsi successivement.
Quelques années avant la guerre Valdés recueillit sous le titre de les
Papiers du docteur Angélique des contes philosophiques et
scientifiques, écrits dans l'intervalle de ses autres ouvrages. La
Guerre injuste qu'on va lire est l'ensemble des articles qu'il publia
dans le grand journal madrilène El Imparcial. Ajoutons enfin qu'une
revue espagnole, Revista quincenal, publie en ce moment un nouveau
roman de notre auteur: Années de jeunesse du docteur Angélique.

Telle est l'œuvre de Palacio Valdés. Quant à l'homme, il est d'une
modestie, d'une bonne humeur, d'une libéralité d'âme, d'une richesse
d'esprit, qui font de sa société un délice. Que ce soit à Madrid, dans
nos Landes où il passe d'ordinaire l'été, il vit seul, lisant beaucoup
ou se promenant. Il n'écrit que s'il lui plaît ou s'il a vraiment besoin
d'exprimer des idées qu'il croit utile de répandre. De là le
retentissement en Espagne des articles qu'il écrivit sur la guerre. Nous
devons à leur auteur la conversion de beaucoup de nos voisins à notre
cause. Qu'il en soit ici publiquement remercié.

\begin{flushright}
ALBERT GLORGET.
\end{flushright}
