%\begin{part}{La Guerre injuste}

\begin{chapter}{La résolution de la France}

La direction de l'Imparcial m'a fait l'honneur de me confier la tâche
d'étudier l'esprit français dans ces moments si critiques.
Quelqu'honneur qu'elle me fasse, je n'aurais pas accepté cette tâche si
des motifs d'ordre moral ne s'étaient d'abord offerts à mes yeux. Je
suis vieux, ma santé est chancelante, j'ai toujours craint le bruit de
la presse. A quoi bon passer du silence au fracas? Pourquoi quitter le
coin où depuis des années, à l'insu de la multitude, je cause à voix
basse avec des esprits épars dans le monde et qui me sont familiers?

Pourquoi? Parce que la voix de ma conscience, cette voix qui, avec les
années, se fait plus forte en tout homme me l'insinue instamment. Alors
que des millions d'êtres humains vivent présentement en Europe, les uns
dans le sang, les autres dans les larmes, a-t-on le droit d'invoquer la
crainte, la maladie, la vieillesse? Laissons la vile matière murmurer;
ce n'est pas l'heure d'écouter ses rébellions. L'heure des plaisanteries
et des aises est passée; il faut maintenant regarder la réalité brutale
bien en face et porter sur les blessures une main pleine de pitié.

\horizontalLine

Me voici donc ici, et il convient à ma sincérité et au respect que j'ai
du lecteur de lui faire ma profession de foi. Je ne suis pas neutre dans
le sanglant conflit qui afflige en ce moment l'humanité; je ne l'ai
jamais été dans aucune dispute qui se soit produite sous mes yeux. J'ai
pu me tromper; mais toujours je me suis résolument placé du côté de
celui qui avait avec lui la raison. Aussi, lorsqu'éclata cette guerre,
ai-je incliné du côté de la France. Car je pensais et je continue de
penser que la raison et la justice sont avec elle.

Durant les longues, les interminables heures de chemin de fer pour
arriver à cette grande ville auparavant si heureuse, si infortunée
aujourd'hui, j'ai eu le temps de faire un minutieux examen de
conscience. Je me suis loyalement demandé s'il n'y avait pas quelque
motif impur dans l'attitude que je prenais en faveur des Alliés.
N'était-ce pas sympathie personnelle? Non; il n'y a pas de pays pour qui
j'éprouve une préférence excessive: je suis persuadé que les hommes sont
partout les mêmes. Il n'est pas, en Europe du moins, de races
supérieures et inférieures: il n'y a que des hommes de bonne ou de
mauvaise volonté. Mon cœur est acquis aux premiers, qu'ils respirent
au milieu des vergers d'Italie ou dans les steppes de Russie. Serait-ce
donc intérêt? Je n'en ai aucun à ce que ce soient les uns ou les autres
qui triomphent. Serait-ce gratitude? J'en dois autant aux deux
belligérants: j'ai reçu de l'un et de l'autre des preuves imméritées
d'estime. Serait-ce par hasard quelque considération politique? Voilà le
motif où il faut s'arrêter. Dans l'ordre politique, en effet, j'admire
l'Angleterre plus qu'aucun autre pays au monde. C'est le pays où l'homme
a pour l'homme le plus de respect, celui qu'on peut appeler aussi en
toute sincérité le plus civilisé. Mais, en revanche, la Russie est le
plus arriéré. Je n'avais donc aucun motif de préférence particulière.

Convaincu qu'en ce moment la mienne est fondée sur la justice, ou sur
ce que j'entends par justice, je suis tranquille et je prends la plume
pour la défendre.

Et maintenant qu'il me soit permis de poser une question. Tous les
germanophiles et tous les francophiles d'Espagne sont-ils descendus
ainsi au fond de leur conscience et se sont-ils sincèrement interrogés
sur les motifs dont ils font la base de leur inclination? Mes
observations ne me permettent pas de l'assurer. Les uns se déclarent
partisans de l'Allemagne parce qu'ils sont autoritaires et mettent la
discipline sociale au-dessus de tout; les autres se prononcent pour la
France parce qu'il s'agit d'une république et qu'ils supposent qu'on y a
plus de liberté qu'ailleurs; les marins sont les amis des Alliés parce
qu'ils admirent la flotte anglaise; les troupes de terre sont en extase
devant les méthodes de guerre allemandes. De candides catholiques
s'écrient: Vive l'Allemagne! parce qu'ils sont sûrs qu'ayant anéanti la
France, le Kaiser n'aura rien de plus pressé que de placer le Souverain
Pontife sur son trône temporel et de rétablir l'Inquisition. Bien des
socialistes, non moins candides, crient: Vive la France! parce qu'ils
supposent qu'après son triomphe la répartition des biens ne se fera pas
longtemps attendre. En général, les violents, les colériques sont avec
les Germains; les pacifiques, ceux dont le cœur est tendre
(bienheureux les tendres de cœur!), penchent du côté des Alliés.

Ajoutez-leur les sceptiques, les frivoles, les capricieux, ceux qui se
prononcent pour les uns ou pour les autres, comme dans une corrida
l'on prend parti pour l'un ou l'autre espada, ou pour tel ou tel
cheval sur le champ de courses.

Et pourtant le litige vaut la peine d'être examiné avec sérieux et
droiture. Le sang de nos frères court en torrents. Nous autres
Espagnols, serions-nous par hasard de tranquilles spectateurs assis au
Colisée pour assister à une fête de gladiateurs? Est-ce que notre
mission consiste à dire quel est celui qui a porté les meilleurs coups
ou mis le plus de grâce à tomber? Non; notre chair saigne en même temps
que saigne celle de nos frères; nos larmes coulent avec les leurs. Nous
ne faisons qu'un devant la justice divine. Demandons-lui de nous
éclairer et de ne pas nous laisser tomber dans l'erreur, afin qu'un jour
elle ne nous demande pas compte de notre injustice.

\horizontalLine

Jamais je n'oublierai l'après-midi du $2$ août $1914$. C'était dans un petit
village des Landes françaises où j'ai l'habitude de passer l'été et
j'étais occupé à regarder un ouvrier qui construisait avec son petit
garçon un poulailler dans mon jardin. Il était 4 heures. Le soleil
nageait dans l'air diaphane; la brise nous caressait doucement les
tempes; les oiseaux marins voltigeaient sur nos têtes. Nous devisions
amicalement, quand tout à coup l'ouvrier s'arrêta de travailler, leva la
tête et s'écria étonné:

--  Monsieur, la cloche!

Je prêtai l'oreille et j'entendis en effet le tintement lointain de la
cloche paroissiale.

--  Y aurait-il le feu?

--  Non, ce n'est pas le feu, répondit-il d'une voix sourde. Et, baissant
de nouveau la tête, il poursuivit sa tâche.

Au bout de quelques minutes il la releva, le visage pâle.

--  Le canon! monsieur.

Je prêtai de nouveau l'oreille, mais je ne parvins point à l'entendre.
Il faut dire que nous nous trouvions à 22 kilomètres de Bayonne.

--  Je n'entends rien.

--  Tu l'as entendu, toi? demanda-t-il à son fils.

--  Oui, je l'ai entendu, répondit l'enfant, plus pâle encore que son
père.

Alors, au loin, un roulement de tambour se fit entendre. Je me sentis
troublé jusqu'au plus profond de mon être. Le tambour! Et son roulement
s'approchait sinistre, fatidique, brisant le silence innocent de la
campagne.

Et sur-le-champ accoururent à ma mémoire les souvenirs de la primitive
histoire de l'humanité. Je revoyais le clan voisin plus nombreux et plus
belliqueux se jeter à l'improviste sur le clan plus faible, s'emparer de
ses troupeaux, violenter ses femmes, égorger ses hommes. Voilà, voilà
les féroces ennemis! Alors aussi le cri d'alarme résonnait dans les
champs; alors aussi les hommes pâlissaient et les femmes serraient les
enfants sur leur sein.

Je compris: une grande nation courait un péril de mort. La patrie de
Pascal, de Racine, de Bossuet, de Rousseau, de Balzac, de Musset, d'Hugo
allait être foulée aux pieds, humiliée, peut-être à jamais anéantie. Ce
n'était pas une guerre romantique comme celle de Napoléon que celle qui
se préparait; il ne s'agissait plus d'un génie ambitieux précipitant à
coups de pied de leur trône de ridicules despotes tenant l'Europe sous
la férule; il ne s'agissait plus d'une incomparable armée courant sur
les pas de son empereur, ivre de gloire, mais non de richesse. La guerre
qui s'approchait était une tragédie sordide, la rumeur d'un peuple qui
vient en rugissant d'envie se saisir des fruits du travail de son
voisin. Peu de mois auparavant les journaux allemands annonçaient
qu'ils exigeraient de la France dans la prochaine guerre une indemnité
de 40 milliards de francs.

Je sortis précipitamment de chez moi et fis presque au pas de course le
kilomètre qui me séparait du bourg. Tous les habitants parlaient entre
eux sans bruit, dans un calme imposant.

Comme je traversais un groupe de femmes, elles fixèrent sur moi un
regard jaloux et hostile. Plus loin, je passai devant un autre: même
effet. J'étais l'étranger qui pénètre, indifférent et curieux, dans une
famille affligée. Pauvres femmes, si vous aviez su que mon cœur était
alors aussi serré que le vôtre!

Je rencontrai ensuite des personnes de ma connaissance: elles
détournèrent les yeux de moi, feignant de ne pas me connaître. Alors,
blessé de cette hostilité, je me dirigeai décidément vers elles.

--  Messieurs, je suis étranger, mais le malheur qui pèse en ce moment
sur vous ne peut pas m'être indifférent. Je suis absolument certain que
vous ne vouliez pas la guerre, que personne parmi vous n'y pensait. Bien
que vous pleuriez, comme de juste, la perte de votre Alsace-Lorraine,
vous n'espériez la recouvrer que par des moyens diplomatiques. Mais on
vous attaque indignement. La justice et la raison sont avec vous. Par
conséquent, je suis, moi aussi, avec vous, et je souhaiterais pouvoir
vous le prouver mieux qu'en paroles.

Ils me serrèrent silencieusement la main. L'un d'eux dit enfin avec
gravité:

--  C'est assez d'humiliations comme cela! Finissons-en une bonne fois!

Et les autres répétèrent chacun leur tour:

--  Il faut en finir, il faut en finir!

Je m'éloignai d'eux et, suivant la route, je revins au bord de la
rivière. Assis dans une barque où il rangeait ses filets, un jeune
pêcheur avec qui j'ai l'habitude de causer m'apparut.

--  Tu as entendu? lui demandai-je en lui désignant l'endroit où sonnait
le tambour.

--  Oui, j'ai entendu. Il faut en finir! me répondit-il sèchement sans
lever la tête.

Je me remis en route et je vis une jeune fille qui vient ordinairement
nous vendre son poisson.

--  Tu vois ce qui arrive? lui dis-je. Tu n'as pas peur?

--  Oui, monsieur, j'ai peur: j'ai deux frères qui doivent immédiatement
partir... Mais il faut en finir, monsieur, il faut en finir!

J'arrivai sur la place et je m'assis à la porte d'un petit café qui se
trouve là. A une table proche, un vieux militaire en retraite disait à
ses amis:

--  Mieux vaut être défait une bonne fois qu'être sans cesse humilié. Il
faut en finir!

--  Il faut en finir! dirent en chœur ses amis.

\horizontalLine

Depuis lors deux années ont passé. Et voici que je reviens en France,
que j'arrive à Paris, et partout, exprimée dans la même forme, c'est la
même résolution qui retentit à mes oreilles: il faut en finir! Oui, la
guerre ne se terminera que lorsque le noir cauchemar qui tourmente la
nation française se sera tout à fait dissipé. Ou la tombe ou la liberté!
Le clan ne se jettera plus sur le clan voisin, tant que ce voisin sera
vivant.

Combien pourtant le timbre des voix est changé! Les voix chantent, les
voix rient, les voix jouent. Un rayon de soleil est tombé sur la
France. On ne baisse plus les yeux; les fronts se lèvent; les regards se
fixent, pleins de lumière, sur notre visage. Un ami me dit gaiement à
l'oreille en m'embrassant sur le quai de la gare:

--  Maintenant c'est sûr!

--  Vous n'avez plus peur que Lohengrin ne paraisse à l'horizon?

--  En tout cas, s'il paraît, ce ne sera qu'avec son cygne.

Voilà où en est venue la France. Voyons maintenant cet optimisme.

\end{chapter}

\begin{chapter}{L'optimisme français}

L'optimisme est à la mode. Il y a aussi des jupes courtes et des jupes
longues dans la philosophie. En ce moment on nous crie de partout à nous
rompre la tête: «Soyez optimistes!». Enfermées dans de jolis livres, ces
voix régénératrices nous viennent surtout d'Amérique. Les psychologues
américains de nos jours ne se lassent pas de répéter cette chanson, qui
est un peu monotone à nos oreilles de Latins. L'un des plus distingués
d'entre eux, Waldo Trine, tonne avec éloquence, dans un de ses derniers
ouvrages, contre l'ennui et la peur, qu'il appelle «les deux noirs
jumeaux». «En attirant à nous par la peur les choses mêmes qui nous
donnent de la crainte, dit-il, nous attirons aussi toutes les conditions
qui contribuent à entretenir la peur dans l'esprit.

Je sais en effet par expérience que la peur est une chose désagréable et
que l'optimisme est bien plus stomacal. Je n'ai cependant jamais trouvé
le moyen intellectuel de se délivrer de la peur. Et si une chose m'a
parfois donné de l'assurance, c'était de voir un couple d'agents de
police près de moi.

Si pour être optimiste, il suffit de vouloir l'être, il me semble qu'il
ne doit pas y avoir une seule personne au monde qui ne le soit. Et c'est
justement ce à quoi prétendent ceux que l'on appelle les «philosophes de
la volonté»: «Soyez optimistes; il n'y a qu'à le vouloir.»

Non, il ne suffit pas de le vouloir. Il est facile à un ténor de donner
le «do de poitrine», facile à un boxeur de porter un bon coup de
poing; mais c'est impossible au reste des hommes. C'est pourquoi dans
son fameux livre The varieties of religious expérience, William James,
le plus remarquable et le plus perspicace de ces philosophes, divise les
hommes en deux classes: ceux qui n'ont eu qu'à naître pour être heureux
et ceux qui pour être nés malheureux ont dû naître deux fois, once born
and twice born. Les premiers sont les optimistes, ceux qui voient tout
en rose. Le monde est régi par des forces bienveillantes qui se chargent
de tout arranger le plus heureusement possible. Le soleil les enchante;
la pluie leur paraît admirable; s'ils se cassent une jambe, ils prennent
cela comme un événement heureux, car ils eussent pu se casser les deux
du coup. A ces optimistes de naissance s'opposent les tempéraments
pessimistes, ceux qui sont possédés d'une tristesse incurable. Pour
ceux-ci, il n'y a point d'événement, si heureux qu'il semble, qui ne
finisse par changer de caractère et se transformer en malheur. Dans
toute joie ils voient un désabusement probable; dans toute fleur, le
ver; dans toute opulence, la faillite prochaine.

Je reconnais qu'on rencontre quelquefois ces deux tempéraments extrêmes,
mais le plus souvent on les rencontre atténués. Ce que je ne puis
cependant admettre, c'est que le premier soit le tempérament idéal,
celui que nous devons tous admirer et souhaiter d'avoir. Ces êtres que
William James appelle «ceux qui sont nés une fois», ce sont des
inconscients, ceux qui ne se rendent pas compte de ce qu'est la vie, de
ce qu'est le monde. En ce sens, l'optimiste par excellence, c'est la
bête, qui ne sait point qu'elle mourra. Mais il est impossible à ceux
qui savent qu'ils mourront d'être optimistes à la façon qu'exaltent les
psychologues américains.

Ne nous faisons pas d'illusions. La vie est âpre, la réalité odieuse. La
faim, le typhus, le cancer, la guerre, sont des hôtes avec lesquels il
faut compter. Qui nous eût dit il y a trois ans que l'Europe civilisée
allait se transformer en troupeaux de tigres et de chacals? Si «ceux qui
sont nés une fois» ne se soucient point de cela, c'est tant mieux ou
tant pis. Pour moi, les vrais hommes, ce sont ceux qui sont «nés deux
fois», je veux dire ceux qui se rendent compte de leur situation sur la
terre, de leur origine et de leur destin immortel. Le premier est le
«vieil homme» de saint Paul, celui en qui dominent les instincts
animaux, celui qui vit tout endormi dans l'inconscience de la nature. Le
second est l'«homme nouveau», celui qui a ouvert les yeux à la lumière,
l'homme spirituel qui s'élève sur son vêtement de chair, comme la
chrysalide pour se muer en papillon abandonne le petit sac qui
l'emprisonnait. «La mélancolie, disait le Père Lacordaire, est
inséparable de tout esprit qui voit loin et de tout cœur qui est
profond, et elle n'a que deux remèdes: la mort ou Dieu». Bénie soit donc
la mélancolie, qui nous révèle notre condition d'hommes. Arrière,
inconsciente allégresse qui nous retient dans les limbes de l'animalité!

\horizontalLine

Dans un des derniers numéros de la Revue des Deux Mondes, le docteur
Emmanuel Labat a publié un article intitulé «Notre optimisme». Il mérite
d'être lu: il est parfaitement écrit, et je le déclare d'autant plus
volontiers que ma façon de penser est diamétralement opposée à la
sienne.

Le docteur Labat est un disciple de la nouvelle école psychologique. M.
James notamment a eu sur lui une influence décisive. Mais M. Labat est
médecin, et comme tel il n'hésite pas, quand il peut, à amener l'eau à
son moulin. Je veux dire qu'il exagère les enseignements un peu nébuleux
et panthéistes de l'école et les transforme, quand il lui est commode,
en enseignements matérialistes.

L'éminent professeur suppose que l'optimisme n'est pas une opération de
l'esprit qui raisonne, mais qu'il vient de plus loin, d'une source plus
profonde et plus intime. L'optimisme, dit-il à peu près, c'est
l'instinct de la vie, l'horreur de la mort, l'allégresse, l'orgueil et
la volonté de vivre.

J'avoue que je ne comprends pas bien cet optimisme qui consiste à avoir
horreur de la mort. Appeler optimisme l'instinct de la conservation est
un abus de langage. Le véritable optimiste doit n'avoir aucune peur de
la mort, puisque nous sommes dans un monde où il faut que l'on meure. Le
martyr chrétien qui allait au supplice en chantant était optimiste,
parce qu'il savait qu'une félicité sans fin l'attendait dans l'au-delà;
de même le musulman qui se jette sur l'épée de l'ennemi parce qu'un
chœur de belles houris l'attend, ou le Chinois qui se laisse
allègrement tuer en Amérique parce qu'il est sûr de ressusciter dans sa
patrie. Quant à celui qui conserve avec inquiétude sa précieuse peau
dans la certitude que quoi qu'il fasse il finira par être la pâture des
vers, celui-là n'est pas optimiste.

Or, c'est de cet instinct de vie, ou de cet instinct de conservation,
comme on disait autrefois, que le docteur Labat fait dériver l'optimisme
français d'aujourd'hui. Il suppose que le Français est optimiste par
nature et que cet optimisme est la sauvegarde de son existence. C'est,
selon moi, une erreur. Il y a en France autant de pessimistes et de
neurasthéniques qu'en aucun autre pays, peut-être même davantage. Et
cela se comprend. Le Français est généralement ambitieux; il aime la
richesse et travaille ardemment pour l'acquérir. Eh bien, dans la
statistique de la neurasthénie, ce sont les hommes d'affaires qui
occupent la première place. Le Français possède en outre un esprit
critique aigu, et un critique n'est jamais optimiste.

Au surplus, j'ai vécu en France durant les premiers mois de la guerre et
je n'ai point observé un pareil optimisme. Ce que j'ai vu, c'est la
résolution, l'inébranlable volonté de se défendre jusqu'à la mort. Et
cela ne peut pas s'appeler de l'optimisme. Au contraire, quand les
Allemands arrivèrent aux environs de Paris, j'ai constaté quelque peu de
dépression et d'abattement. Mais, et je me plais à le déclarer, la
ferme et courageuse résolution des Français n'en fut nullement altérée.

Puis ce fut la bataille de la Marne. L'esprit français s'exalta soudain;
un chaud optimisme régna quelque temps. On crut à la victoire immédiate,
on pensa même conquérir l'Allemagne et entrer à Berlin. Mais des mois
passèrent et l'on en vint à se dire qu'il ne fallait pas s'attendre à
cette sorte de victoire. Le Français est le raisonneur par excellence.
Peut-être en d'autres pays les hommes témoignent-ils de qualités plus
hautes. Mais le bon sens est le patrimoine de la France, sauf lorsqu'on
touche à sa vanité nationale, car elle a vite fait alors de passer les
limites de la raison. Il est vrai qu'elle sait y revenir promptement et
s'accommoder des circonstances avec une étonnante facilité.

Bien des personnes ont pensé toutefois qu'il serait possible aux
Français de percer les lignes allemandes, de recouvrer le terrain perdu
et d'avancer sur le territoire ennemi. Dans les derniers jours de
septembre, un sergent arriva dans le village que j'habitais. C'est un de
mes grands amis. Il exerce la profession de notaire, mais par
tempérament c'est un soldat: il est énergique et courageux.

--  Quand percerez-vous donc les lignes? lui demandai-je souriant.

--  Quand nous le voudrons, me répondit-il avec tranquillité.

--  Vous parlez sérieusement?

--  Oui, sérieusement. Nous attendons seulement qu'on nous en ait donné
l'ordre.

Cet ordre arriva peu de jours après et l'on sait ce qui s'ensuivit. Au
prix de sacrifices énormes, d'une quantité de sang prodigieuse, on
avança de trois kilomètres. Au moment où j'écris, la même chose arrive
aux Allemands, avec encore moins de bonheur.

Aujourd'hui l'optimisme a changé de direction. Si l'on veut savoir ce
que c'est que de calculer, il faut venir en France. Un de mes amis m'a
prouvé il y a peu de jours, le crayon en main, que les empires centraux
ont tels et tels moyens de défense, tant de réserves métalliques, qu'ils
peuvent tenir jusqu'à telle époque et que, cette époque passée, ils
devront succomber. Les Français considèrent l'Allemagne comme une place
assiégée. Elle ne sera point prise d'assaut, mais elle tombera rendue de
faim. Ils ont dans la victoire une confiance aveugle, absolue.

\horizontalLine

Mais cela n'est pas de l'optimisme, dira le docteur Labat. Il s'agit là
d'un calcul, de la solution d'un problème, et l'instinct vital n'a rien
à voir là-dedans. Cependant, pour moi, c'est cela qui est le véritable
et légitime optimisme, car il procède de la raison. L'autre, qui vient
du fond même de notre nature animale, pourra nous rendre parfois la vie
plus douce, plus légère; mais il est extrêmement dangereux. Si l'on veut
bien tourner les regards en arrière et se rappeler l'histoire des
personnes que l'on connaît, tout le monde y trouvera quelque grande
catastrophe ou tout au moins une succession de contrariétés produites
par cet optimisme instinctif.

A cette heure donc, les Français s'occupent à faire des calculs. Ils ne
disent pas toutefois ce qu'on lit au fond de leurs yeux. Leur calcul le
meilleur, c'est qu'ils comptent sur leurs bras et sur leur tête. Et, de
même que le plus habile marin du monde est l'Anglais, le Français est le
meilleur des soldats. Cela n'a rien d'étonnant: cent ans à peine le
séparent de ces autres soldats qui parcoururent toute l'Europe en
vainqueurs. Les traces de l'hérédité ne s'effacent point en cent ans.

    Où le père a passé passera bien l'enfant

disait Musset.

Au reste, ne parlons pas de la valeur. Russes, Allemands, Français,
Bulgares, tous se sont également bien battus. Mais il y a pour le soldat
d'autres qualités d'une importance capitale: la ruse, l'allégresse,
l'habileté manuelle, l'improvisation. Depuis les temps de Jules César,
la race des Gaulois s'est toujours distinguée par ces qualités mêmes. Le
Gaulois est un homme fertile en recours. Vous le verrez louer une maison
à moitié démolie, image de la désolation; mais repassez par là quelques
mois plus tard, et vous serez tout surpris de trouver un nid
confortable, entouré de fleurs. Cuisine, jardin, peintures, terrasse: il
aura tout improvisé.

Un de mes voisins de campagne, dans les Landes, avait besoin d'un
garage. Il vit venir un maçon, qui lui en construisit un en quelques
jours et d'une façon parfaite. Peu après, ce maçon se trouva sans
travail. Mon voisin cherchait alors un jardinier; le maçon lui offrit de
remplir cette charge, et il s'en acquitta avec une intelligence dont
nous fûmes tous émerveillés. Plus tard mon même voisin vint à manquer de
cuisinière. Le maçon passa à la cuisine et il y fut un cuisinier
admirable.

-- Pour Dieu, dis-je à mon voisin, n'allez pas congédier votre nourrice:
je vois déjà votre homme donner le sein à votre fils!

La France est pleine de ces hommes-étuis. Or, dans une guerre longue
comme celle-ci, ils sont d'une grande utilité. Les Allemands mettent
toute leur confiance dans leurs machines; mais la meilleure de toutes
les machines, c'est l'homme. Avec du talent, la plus petite force
devient formidable. Les Allemands sont supérieurs par le nombre, par la
préparation, par les machines de guerre; mais les moyens des Français,
c'est eux-mêmes, leur adresse et leur sang-froid. Les Allemands ont plus
de canons et de plus gros; mais les artilleurs français pointent et
dissimulent les leurs plus adroitement. Ceux-là possèdent de splendides
cuisines roulantes; mais, avec de pauvres feux de campagne, ceux-ci
mangent mieux.

Joffre est l'incarnation de cet esprit gaulois, fait d'astuce, de
courage, de prudence et de gaieté. C'est lui qui a sauvé la France au
moment suprême par sa tactique admirable; c'est lui qui, patient et
énergique, attend que le fruit soit mûr pour secouer l'arbre; c'est lui,
homme de pitié, que les soldats appellent «le père Joffre», parce qu'il
est avare du sang de ses fils. Louange à ce Gaulois insigne qui fut le
boulevard choisi par la Providence pour sauver la civilisation latine
et l'indépendance des peuples faibles! Le jour où sa statue se dressera
sur une place de Paris, nous irons tous, non point pour y planter des
clous, mais pour la couronner de fleurs.

Il ne ressemble pas aux généraux allemands, qui, eux, ont non seulement
copié strictement la tactique de Napoléon, mais aussi ses procédés
impitoyables. «-- Sire, Sire, disait le général Junot à l'Empereur, il
est absolument impossible de s'emparer de cette batterie autrichienne:
un feu d'enfer balaie tous les hommes.-- Avancez! répondait
Napoléon.-- Chaque régiment qui avance est un régiment perdu.-- Avancez!»
répétait Napoléon.

Il importe de ne pas confondre le peuple allemand avec ceux qui le
dirigent aujourd'hui politiquement et militairement. L'allemand est un
peuple doué de solides vertus: il est courageux, intelligent,
opiniâtre, laborieux, idéaliste. Mais, comme tous les idéalistes, il
manque d'esprit critique et c'est pourquoi il obéit facilement à tout ce
qu'on lui suggère. Sa race lui est montée au cerveau et c'est ce qui lui
a fait dire et commettre un assez beau nombre de sottises. Néanmoins,
tout le monde s'accorde à reconnaître ses hautes qualités. Mais ces
qualités ont une tache, la jalousie des Anglais: jalousie de parents,
qui se dissipera bientôt.

Aussi est-il intolérable, extrêmement pénible, d'entendre M. Maurice
Barrès appeler les Allemands «sale race». Tous les hommes de bon sens en
France ont réprouvé ce langage, et la Presse, la première.

Pourtant le docteur Labat lui a donné l'appui d'arguments médicaux. Il
dit que l'instinct de vie (encore, l'instinct de vie!) justifie de
pareilles injures, qu'il a pris l'avis des blessés de son hôpital et
qu'ils sont unanimes à donner raison à M. Barrès et à reconnaître que
lorsqu'on porte un coup de baïonnette en s'écriant: «Tiens cochon!
Crève, sale bête!», la baïonnette fait quelques pouces de plus dans le
corps de l'ennemi.

Je confesse que des raisons chirurgicales de cette sorte ne m'ont point
convaincu. Ma pensée vole vers cette mémorable bataille de Fontenoy, où
le général français se découvre et crie en s'approchant de l'ennemi:
«Messieurs les Anglais, tirez les premiers!» Aujourd'hui ce mot peut
paraître don-quichottesque; mais entre le «tirez les premiers» du
général et le «crève, sale bête!» de M. Barrès, je n'hésite pas à
préférer le premier. On peut être sûr que celui qui dit «tirez les
premiers» ne tournera jamais le dos à l'ennemi; quant à l'autre, on n'en
peut rien assurer.

Quels vilains temps que ceux où nous sommes! C'est dans les vôtres,
nobles hommes, que j'eusse aimé vivre et non point en ceux, sans
honneur, où l'on conseille aux soldats de se salir les lèvres pour se
donner du cœur et où l'on commande aux officiers de fusiller les
femmes et de jeter des bombes la nuit sur des berceaux d'enfants.

\end{chapter}

\begin{chapter}{Méditation sur le conflit}

Ni les gaz asphyxiants que dégagent les tranchées allemandes, ni la
rhétorique, plus asphyxiante encore, dont les Germains et les
germanophiles se servent pour exalter leur morale, n'arriveront à
étouffer la vérité rebelle.

Cette vérité, c'est que cette guerre monstrueuse à laquelle l'humanité
assiste étonnée a été longuement méditée, préparée, puis déchaînée par
une nation européenne dans le seul but de dominer matériellement et
moralement les autres.

Et comme cette vérité saute aux yeux et qu'il est impossible de la nier,
les Espagnols qui sympathisent avec cette nation croient justifier leur
sympathie en rappelant les torts que les Français et les Anglais nous
firent en des temps plus ou moins anciens. Ainsi le loup de la fable
évoquait pour manger l'agneau les mauvais traitements qu'il avait reçus
de ses pères.

Dans tous les temps et sur tous les points du globe habité, c'est contre
leurs voisins que se sont battus les peuples et non pas contre ceux qui
vivaient au loin. Il est bien probable que si Berlin était à la place de
Bordeaux ou de Lisbonne nous en serions déjà venus aux mains avec les
Allemands, comme nous l'avons fait avec les Portugais et les Français.
L'Allemagne et l'Autriche, qui sont non seulement des voisines mais des
sœurs, ont été en guerre de nos jours même.

Quand on sort du terrain de la haine et qu'on passe sur celui des
raisons, les arguments se présentent sous les formes les plus diverses.

Contre l'Angleterre, on se sert de l'argument chrématistique;
l'Angleterre a de très riches colonies, des territoires immenses dans
les cinq parties du monde, tandis que l'Allemagne, pays hautement
civilisé et tout aussi méritant que la Grande-Bretagne, possède peu de
chose hors de chez elle. Pourquoi?

Ceux qui s'indignent d'avoir à poser cette question sont le plus souvent
de riches propriétaires. Ils ne se rendent pas compte que le langage
qu'ils tiennent contre l'Angleterre est justement celui que tiennent
contre eux-mêmes les socialistes et les communistes. «Nous valons autant
que vous, disent-ils. Mais, tandis que vous êtes riches, nous sommes
pauvres: pourquoi? Vous êtes des voleurs, livrez les biens que vous
possédez injustement.»

L'argument n'aurait de portée que si la Grande-Bretagne était incapable
de coloniser. Ses colonies seraient-elles plus heureuses entre les mains
de l'Allemagne? C'est à ces colonies qu'il faudrait le demander.

Contre la France, c'est de l'argument religieux qu'on se sert. Cette
nation qui a décrété la séparation de l'Église et de l'État et chassé
les ordres religieux, mérite un châtiment exemplaire.

Personne ne l'a rendue responsable des sanglants excès de la Convention,
ni des assassinats commis par Robespierre et Marat. Pourquoi l'accuser
aujourd'hui des dispositions d'un ministre anticlérical?

En admettant d'ailleurs que l'argument fût juste, ce qui ne le serait
certainement point, ce serait de l'étendre à ceux qui n'ont commis
aucune faute. La masse du peuple en France est en effet catholique et
c'est de son plein gré, sans le moindrement recourir au trésor public,
qu'elle soutient aujourd'hui le culte catholique avec la même décence
qu'autrefois.

On oublie ou l'on feint d'oublier que c'est de cette France impie que la
pensée chrétienne rayonne à travers le monde une lumière merveilleuse.
Non seulement il y existe en ce moment un groupe de philosophes
spiritualistes, dont Boutroux, le chef, livre sur le terrain de la
pensée de glorieuses batailles aux savants matérialistes allemands comme
les Wundt, les Haeckel et les Ostwald; mais il y existe aussi une
phalange d'éminents apologistes catholiques, des prêtres le plus
souvent, dont les livres font la consolation de tous les croyants de
l'Europe. On oublie que quelques-uns de ces prêtres se battent
aujourd'hui dans les tranchées de l'Alsace et des Flandres, et qu'ils
s'étonnent et s'affligent d'entendre les reproches que font à leur
patrie ceux qui se donnent pour les hérauts de la chrétienté.

Contre la Russie, c'est de son retard qu'on tire un argument. Ces
pauvres Russes! Ils n'ont point de canons de précision, point de chemins
de fer stratégiques, point de gaz asphyxiants; ils mangent avec les
doigts: ce sont de vrais sauvages. Il faut aller leur apprendre le
maniement des armes à feu et de la fourchette.

Pourtant, ces sauvages, qui sont armés de massues de fer en guise de
fusils, à en croire les journaux allemands, ces sauvages-là se battent
depuis longtemps contre toute l'armée autrichienne et plus d'un tiers de
l'armée allemande.

Contre la Belgique enfin, on use d'un argument sanchopancesque. Qui donc
a fourré la Belgique dans une si folle aventure? Comment a-t-elle eu
l'audace de faire front au colosse allemand? Ne sait-elle pas que rien
n'est plus prudent que de rester en bons termes avec les forts? Si elle
avait laissé tranquillement passer les armées du Kaiser, elle ne serait
pas dans la calamité où elle se trouve, elle aurait reçu une pleine
bourse de pièces d'or et qui sait? à la fin de la guerre, peut-être un
petit morceau de la France.

Voilà ce que l'on entend ici. Là-bas, en Allemagne, on méprise les
raisons: nous entrons sur le théâtre de la volonté rugissante et de
l'automatisme. Un seul mot nous en vient «Nous voulons!» Et de toutes
les régions du monde où la volonté l'emporte sur la raison, les hommes
répondent à ce «nous voulons»: «Puisque vous le voulez, nous le voulons
aussi.»

C'est un cas de désagrégation mentale dans lequel le psychisme
inférieur, le centre de l'automatisme, brise son engrenage avec la libre
raison et s'abandonne passivement à toutes les fantaisies de
l'hypnotiseur. Les hypnotiseurs du peuple allemand, ce sont les magnats
de la politique et de l'armée prussienne, secondés par la poltronnerie
de quelques intellectuels. Ce sont eux qui ont imposé à ce peuple et la
guerre et la férocité dans la guerre. Ils lui ont dit: «Gardez-vous de
votre cœur comme d'un ennemi; fusillez des prêtres, démolissez des
monuments, violentez des femmes, asphyxiez les enfants, essayez de tous
les moyens pour atterrer l'ennemi.» Et ces honnêtes citoyens, ces bons
pères de famille que nous avons tous connus fusillent, violent, saquent,
asphyxient. Si on leur disait en outre de sacrifier les prisonniers, ils
les sacrifieraient aussi.

Un pareil état de misère morale inspire plus de pitié que de haine. Ce
sont des hommes en sommeil; ce n'est pas à eux qu'il faut imputer leurs
horreurs, mais à ceux qui les ont ainsi magnétisés.

A qui donc enverrons-nous le compte de la dispersion qui s'est produite
dans les centres cérébraux de quelques-uns de mes compatriotes? Car il
y a parmi nous des individus qui rougissent dès qu'on prétend que les
Teutons n'ont pas bien fait de livrer Louvain au pillage et de fusiller
des prêtres; ils rougissent, se grattent la tête, sentent bouillir leur
cervelle et finissent par s'écrier qu'ils en auraient fait tout autant,
qu'ils auraient tué plus de prêtres encore et qu'ils en auraient même
ensuite mangé en sauce tartare.

J'ai eu l'horreur d'entendre des dames se féliciter du torpillage du
Lusitania et des exploits des zeppelins.

Le naufrage du Lusitania est une chose effroyable, mais ce naufrage de
l'âme féminine est plus effroyable encore...

Comme tout ce qui écorche un instant la croûte de notre malheureuse
planète, cette guerre aura sa fin. L'épais nuage qui couvre aujourd'hui
toute l'Europe se dissoudra enfin dans l'atmosphère azurée; la terre
maternelle boira le sang, dévorera les os et, dans son sein fécond, la
vie immortelle poursuivra son travail mystérieux; les prés auront de
nouveau des fleurs, les arbres agiteront de nouvelles branches à la
brise du soir, les oiseaux de Dieu se remettront à bénir de leurs
trilles le lever de l'aurore.

Et que restera-t-il de tout cela? Une grande honte et un grand remords.

Oui, un grand remords.

Un jour viendra (le Ciel nous le donne bientôt!) où ces automates
assassins de femmes et d'enfants sortiront de leur stupeur hypnotique.
Épouvantés d'eux-mêmes, ils tomberont alors aux pieds de leurs fils et
leur demanderont pardon de les avoir tant scandalisés, d'avoir outragé
sous leurs yeux d'enfants l'honneur du genre humain, d'avoir voulu leur
arracher du cœur la seule chose pour laquelle l'homme puisse vivre et
doive mourir.

\end{chapter}


\begin{chapter}{La stratégie de Napoléon}

Je suis allé à Marly et à la Malmaison. On éprouve un plaisir physique à
ne plus entendre le bruit de la métropole et à passer quelques instants
dans la fraîcheur et la tranquillité des champs. Mais le plaisir est
encore plus vif pour l'esprit, surtout quand l'endroit où l'on est vous
offre son passé comme un refuge contre un présent douloureux. Vus de
loin, et lorsqu'ils sont déjà à demi ensevelis dans l'abîme du temps,
les événements les plus pénibles allègent l'âme au lieu de l'affliger.
C'est là le secret de l'art. Le monde, comme pure représentation, ne
fait jamais de mal.

Il n'y a point trace à Marly de la cour fastueuse qui y vécut. Marly est
un tranquille village où l'on entend battre la faux et mugir des
troupeaux. J'en ai parcouru les bois et les prairies avec respect,
évoquant la figure du Roi-Soleil, qui se plaisait tant dans ces lieux.
Son amour excessif pour Marly servit de prétexte à un de ses courtisans
pour dire, dans un transport d'adulation, que «la pluie de Marly ne
mouillait point». Louis XIV avait le gosier large, mais il ne put avaler
cette bouchée-là.

La Malmaison me fut malheureuse: la guerre a fait fermer le palais.
Gardiens et cicerone sont sous les armes. Je dus me contenter de longues
promenades dans le parc et de mes souvenirs du vainqueur d'Austerlitz.

Louis XIV et Napoléon! Deux monstres d'égoïsme et d'orgueil. Saint-Simon
a analysé l'orgueil du premier avec une sagacité merveilleuse; Taine,
celui du second. Mais, quoi! j'ai connu une couturière qui était aussi
égoïste que Napoléon et un cireur non moins vaniteux que Louis XIV.

Pour moi, je crois que si nous prenions un passant au hasard de la rue
et que nous lui infusions le courage et l'intelligence de l'Empereur, je
crois bien qu'on en ferait un autre Napoléon. En tout cas, il ne serait
pas en reste pour l'égoïsme. Et si nous le dotions du pouvoir de Louis
XIV, ce serait un autre Louis XIV, et ce n'est probablement pas
d'orgueil qu'il manquerait non plus. Égoïsme et orgueil nous viennent
ensemble et tout naturellement, et ceux qui s'en délivrent sont des
êtres exceptionnels devant qui l'on devrait s'agenouiller.

Que de souvenirs dans cette Malmaison! Derrière chaque massif de fleurs
la gracieuse figure de l'impératrice Joséphine semble nous sourire.
Elle y fut heureuse, et puis la plus infortunée des femmes. C'est là
que, victime de l'implacable égoïsme de son mari, cette douce et
sympathique créature rendit son âme à Dieu. Toutes les idylles de ce
monde misérable se terminent dans les larmes.

Et ma mémoire s'emplit soudain de ces jours dramatiques où Napoléon
rentre à Paris avec la résolution secrète de répudier sa femme. Il est
d'abord plus cérémonieux et plus froid avec elle; il ferme ensuite toute
communication entre leurs appartements; il lui fait connaître enfin sa
décision par des émissaires diplomatiques.

Que devait-il se passer dans le cœur de cette noble femme quand elle
constatait que l'homme idolâtré, que l'homme qui lui avait donné avec
son amour le plus haut trône du monde, allait rompre le sacré, le doux
lien qui les unissait, et partager son lit et sa gloire avec une autre?
Je crois vraiment que c'est alors que fut signé dans le ciel la sentence
qui condamna l'Empereur. Malheur à qui maltraite un enfant ou brise le
cœur d'une femme! Les anges ne tardent pas à se venger de lui.

Je ne voudrais pas que l'on prît cela pour des niaiseries. Qui peut dire
qu'à la balance divine une larme ne pèsera pas plus qu'un empire? Le
monde n'est que le symbole d'une réalité plus haute. Un mot tombé des
lèvres d'un humble charpentier de Nazareth a fait trembler la Création.
Des chevaux, des batailles, des canons, cela n'est rien; les empires
sont des ombres, les étoiles des apparences, la gloire un songe. Mais la
parole d'un homme bon subsiste éternellement.

Les milliers d'êtres que Bonaparte a sacrifiés à son ambition ne
déposeront pas tous contre lui au jugement dernier. Beaucoup étaient
tout aussi ambitieux, tout aussi avides de gloire que lui. S'ils y ont
perdu la vie, il exposait aussi la sienne à tout instant: c'est qu'alors
on ne se battait pas de loin comme de nos jours. Mais quand sonnera
l'heure de la justice suprême, l'impératrice Joséphine se dressera et
lira sanglotante au Conseil le renoncement de ses droits, et l'Empereur
sera irrémédiablement condamné.

Napoléon était un homme de proie. Je répète que nous le sommes tous
quand on nous pourvoit de griffes convenables. Il s'est laissé pousser
par cette loi d'ascension qui régit la vie, par ce que l'on appelle
aujourd'hui «la volonté de puissance».

Il y a dans chaque homme un tyran qui se sert de ses moyens pour courir
et bousculer, comme une automobile de sa gazoline. C'est le Destin des
anciens, la fatalité des modernes. Napoléon croyait aveuglément au
destin. «La politique, voilà la fatalité», disait Gœthe dans la
courte entrevue qu'il eut avec l'Empereur. Et ce disant, ses yeux
exprimaient la tristesse et l'inquiétude. Tous les hommes tremblent,
même les plus grands, lorsqu'ils parlent du destin; car ni le caractère,
ni le courage, ni la prudence ne peuvent rien contre lui. Il n'y a qu'un
être au monde qui soit capable de mépriser le destin: c'est le saint. Si
l'on avait parlé de fatalité à sainte Thérèse ou à saint Vincent de
Paul, ils se seraient mis à rire.

L'art de la guerre avait besoin d'un maître; tous les arts en ont eu.
Alexandre, César étaient loin; leur stratégie ne valait plus rien pour
le monde moderne. Bonaparte vint, et il trouva tout prêt: poudre,
fusils, et des hommes pareils à des Romains, enthousiastes de leur
grandeur et ayant du sang de trop dans les veines.

Je me suis attaché à étudier l'histoire de ce grand séducteur de la
jeunesse et je n'y ai point trouvé les magnifiques projets qui lui sont
attribués, et qu'il s'attribuait, se trompant peut-être lui-même: la
résurrection de la puissance romaine, la restauration de l'Empire de
Charlemagne, etc. Je n'y ai vu qu'un grand amateur, un homme passionné
de l'épée, comme Michel-Ange avait la passion de l'ébauchoir, Rubens
celle du pinceau, Balzac celle de la plume. Il ciselait, peignait sur le
champ de bataille. La guerre n'était pas pour lui qu'un moyen, c'était
aussi une fin. Il en tirait son plaisir le plus fort et c'est pourquoi
il ne voulut pas l'abandonner quand il en était temps encore, et se
perdit.

Le culte de Napoléon, comme le culte de Bouddha, n'a pas laissé de
profondes racines dans le sol où il est né. Ainsi en fut-il d'ailleurs
de notre religion, qui, née en Orient, germa et se propagea en Occident.
Quand les vétérans qui l'avaient suivi dans ses romantiques expéditions
furent morts ou dispersés, l'hostilité commença. Des dards vinrent de
partout se planter dans la statue du grand homme: il en vint des hauts
sièges remplis par les conservateurs aussi bien que de la jeunesse
généreuse, il en vint des ignorants comme des intellectuels. Puis, les
idées pacifistes et humanitaires se développant en France, la
désaffection se manifesta de plus en plus. Les origines de la France
contemporaine de Taine sont l'expression la plus vive de cette
désaffection. Là le héros merveilleux n'est plus qu'un heureux
aventurier, un condottiere dépourvu de sens moral, de grandeur et de
poésie.

Lorsqu'il fut à peu près abandonné des Français, le culte de Napoléon se
réfugia en Allemagne. Les Allemands, qui ont de nombreuses et grandes
qualités, ne brillent point par l'originalité. Comme les Japonais,
c'est un peuple d'adaptation et non d'invention. A peine lui doit-on
quelques-unes des découvertes modernes. Mais il sait admirablement se
servir de ce qu'ont découvert les autres et porter ces découvertes à
leur plus grande perfection. Les Anglais et les Français ont plus de
génie inventif; les Allemands l'emportent dans la façon d'opérer.

S'il est un peuple sur terre qui a mérité la palme de l'invention, c'est
le peuple anglais. Non seulement il a trouvé des méthodes et des
facilités dans les arts industriels, mais il en a trouvé même dans la
façon de vivre. Et cette façon de vivre, ils l'ont peu à peu imposée au
monde entier, avec leurs plus extravagants caprices. Cela tient au
respect qu'on a en Angleterre pour l'initiative individuelle. Il y a
aussi en France une habileté naturelle; elle n'est pas accumulée en
quelques géants, mais éparse dans tous les esprits et dans toutes les
mains. C'est une chose bien connue: les Français sont aptes aux choses
les plus diverses.

En Allemagne, au contraire, l'initiative privée existe à peine; les
Allemands tirent toute leur force de la discipline et de la patience.
Tacite disait des Germains qu'ils n'étaient capables que des grands
efforts, mais que la continuité du travail les impatientait. Ce coup-là,
le grand historien n'a vraiment pas mis dans le mille; c'est précisément
la patience qui est le trait caractéristique de l'Allemand. Il y a
quelques années, un professeur de collège Allemand me disait que les
petits Espagnols étaient d'ordinaire mieux doués que les petits
allemands, mais qu'à la longue, par la constance dans l'effort, ces
derniers ne manquaient jamais de les surpasser.

Il n'est donc pas étonnant qu'ayant perfectionné la vapeur,
l'électricité, l'aviation, ils aient fait merveilleusement avancer
l'art de la guerre. Pour l'étudier, ils sont accourus à la source la
plus pure et la plus abondante, à la stratégie de Napoléon. A ce point
de vue-là, l'Empereur est sans doute le plus grand maître qui ait
existé, et peut-être le plus grand qui sera jamais. La guerre n'avait
aucun secret pour lui. Il enfermait dans son esprit une telle somme de
pénétration, de décision et surtout de sens commun, qu'il en était
invincible.

C'est que la stratégie a été et sera toujours une question de bon sens:
elle ne peut pas évoluer. Le maréchal allemand chef d'état-major
Schlœffer a écrit un livre pour démontrer que la bataille de Cannes,
livrée par Annibal, est le modèle ou l'idéal des batailles. Quelles
qu'elles soient, le seul but qu'y poursuit une armée ne peut être et ne
sera jamais que l'enveloppement de l'ennemi.

Pendant la seconde moitié du dix-neuvième siècle, les stratèges
allemands se vouèrent tout entiers à l'étude des guerres
napoléoniennes. Le nombre de livres et d'articles de revue qui ont paru,
de conférences qui ont été faites sur ce sujet, est incalculable. On
apprit les batailles par cœur, on pénétra jusqu'aux replis la pensée
du maître. En $1870$ les Allemands ont appliqué avec le plus heureux
succès le système de convergence ou de concentration des forces que
Napoléon employa dans toutes ses premières campagnes et surtout dans la
campagne d'Italie. Dans cette guerre-ci, les Allemands ont été empêchés
par les circonstances de développer cette méthode en grand; mais ils ont
eu recours à celle dont Napoléon dut se servir dans la campagne de $1813$.

La situation des armées allemandes aujourd'hui est presque exactement la
même que celle qu'occupaient alors les armées de Napoléon. Entouré par
les Alliés de cette époque, il s'appuyait avec le meilleur de son armée
sur le centre de l'Allemagne, près de Dresde. Il avait dans le Nord,
pour s'opposer à celle de son ancien subordonné Bernadotte, une armée
dite armée de Berlin; à l'est, une autre armée dite armée de Silésie
devait résister à celle que commandait le maréchal Blücher; au Sud enfin
une troisième armée faisait face aux Autrichiens et aux Prussiens du
maréchal de Schwarzenberg. Sa tactique consistait dans un mouvement de
va-et-vient, ce que l'on appelle à présent «jeu de navette». Il ajoutait
soudain ses forces à celles d'une des armées de la périphérie, puis à
une autre, à son gré. La tactique des Alliés se bornait à se retirer
quand l'Empereur accourait d'un côté et en même temps à s'avancer de
l'autre.

Ce mouvement de va-et-vient, ce jeu de navette, c'est ce que font en ce
moment les Allemands, avec des moyens infiniment plus efficaces, en
transportant leurs forces de l'Orient à l'Occident et inversement.
Napoléon exécutait ces mouvements à marches forcées; ils s'accomplissent
aujourd'hui en wagons ou en automobiles. Napoléon les dirigeait
lui-même, c'est aujourd'hui le soin d'un état-major, sous la direction
du général en chef.

Les Alliés de $1813$ réussirent enfin à serrer le cercle et obligèrent
Bonaparte à livrer la bataille de Leipzig. Il y fut défait, et c'est
miracle qu'il ait pu sauver son armée et porter en France le théâtre des
opérations. Les Alliés d'aujourd'hui obtiendront-ils de réduire le
cercle allemand et forceront-ils l'ennemi à accepter la bataille avec
des forces inférieures aux leurs? C'est le secret de l'avenir.
L'Angleterre l'a prévu, et elle déploie aujourd'hui contre l'Allemagne
le même plan, le même système dont elle s'est servi obstinément pour
abattre Napoléon.

Si, contre toute vraisemblance, les Allemands venaient à vaincre, les
Français auraient alors tout à la fois la satisfaction et la peine
d'avoir été battus par le chef même à qui ils doivent leur plus grande
gloire militaire.

\end{chapter}


\begin{chapter}{Les socialistes français}

Il n'y a pas d'homme avec le cœur en place qui ne se soit quelquefois
senti socialiste. Il suffit de descendre dans une mine, de rencontrer à
la porte d'un théâtre quelque mendiant transi de froid et de faim, pour
qu'entre en branle la corde de nos raisonnements habituels et que nous
nous rendions compte que nous sommes tous un peu fourbes et que nous
marchons sur un terrain mouvant.

Et il y a pourtant des individus qui, au seul mot de «socialisme»
prennent l'air navré, se grattent la tête et lancent d'odieux sons
gutturaux; quelques-uns versent des larmes abondantes. Des bombes
éclatent semant l'extermination, des mains noires qui fouillent leurs
archives, d'autres mains, plus noires encore, qui forcent leur tiroir,
des imprécations, des blasphèmes: tout cela se lève devant eux en une
vision terrifiante.

Il n'y a pas de quoi. Comme le mot l'indique, le socialisme ne signifie
rien d'autre au fond que désir et résolution d'organiser la société
d'une façon plus juste. Ce désir et cette résolution sont parfaitement
légitimes. A moins que nous ne nous imaginions que la société ait
atteint la perfection.

Mais si ce désir est mêlé de haine, tout faiblit et tombe. La haine est
le dissolvant le plus efficace qui soit au monde. Dès que ce dieu
infernal fait son apparition, tout change d'aspect et s'assombrit. Et
c'est malheureusement en compagnie d'une divinité si funeste que le
socialisme a paru de nos jours.

Un leader du socialisme espagnol que je rencontrai dans une fonda,
il y a quelques années, me disait: «Détrompez-vous. Cette affaire se
résoudra comme elles se résolvent toutes ici-bas, par la force. Je lui
répondis: «Mon cher, je crains que vous ne soyez dans l'erreur. Cette
affaire comme toutes celles d'ici-bas, se résoudra par l'amour.»

Le temps commence à me donner raison. Qui peut s'imaginer aujourd'hui
qu'une révolution populaire vienne à triompher, alors que la bourgeoisie
dispose de mercenaires avec des mausers, des canons à tir rapide et des
mitrailleuses?

Oui, l'amour. C'est le sentiment de fraternité guidé par la raison qui
se chargera de résoudre ce problème, en limant peu à peu les irritantes
inégalités sociales. La Nature ne procède pas par bonds, mais la société
non plus. La rive est loin, mais elle est plus près que nous ne le
pensions naguère.

Le socialisme moderne a sa force en Allemagne. C'est une affirmation qui
étonnera et chagrinera ceux de nos germanophiles qui ne peuvent pas se
figurer qu'il nous vient d'Allemagne autre chose que la discipline,
l'autorité, la soumission. Et après tout, ils ont raison. Les masses
socialistes sont beaucoup plus disciplinées en Allemagne que partout
ailleurs. Aussi sont-elles beaucoup plus dangereuses. Cette discipline
tuera l'autre.

En France, le socialisme a toujours été plus théorique que pratique. Il
y eut diverses classes de rêveurs. Les uns s'attaquèrent à la propriété:
ce furent les communistes. Les autres attaquèrent la famille: ce furent
les fouriéristes, ceux du fameux phalanstère. D'autres, la religion: ce
furent les saint-simoniens. Cependant aucun de ces rêveurs n'a réussi à
entraîner et à soulever les masses. Aucun n'a été capable d'organiser
une manifestation de 300.000 hommes à Paris, comme cela s'est produit à
Berlin, il y a quelques années.

Si vous veniez en France et que vous parcouriez les provinces, vous
seriez surpris d'apprendre ce que sont les hommes qui représentent
aujourd'hui le socialisme. Dans un village, vous voyez un joli jardin
remarquablement soigné et entouré de grilles; au fond, un hôtel
magnifique; des jardiniers arrosent, taillent; sur la terrasse, de
jeunes domestiques gracieusement vêtues, tablier blanc et coiffe
blanche. «A qui cette propriété?» demandez-vous? «A M. F..., vous
répond-on; le chef du parti socialiste d'ici.» Vous allez chez un
médecin fameux pour le consulter. Un domestique en livrée vous ouvre la
porte; la maison est tenue avec un luxe extraordinaire; au moment d'être
introduit dans le cabinet, vous jetez un coup d'œil dans la salle à
manger et vous apercevez une nombreuse famille qui prend le thé. Ce
médecin, c'est le fameux B..., directeur-propriétaire d'une revue
socialiste. Vous entrez dans une église pour entendre la messe et en
sortant vous rencontrez un monsieur qui attend une dame. Habillée avec
une suprême élégance, son livre de prières à la main, la dame rejoint le
monsieur, souriante, lui passe son livre, lui prend le bras et ils
s'éloignent en devisant gaiement. C'est M. D..., le député socialiste de
la région.

Il semble bien que ces socialistes français ne soient dangereux ni pour
la propriété, ni pour la famille, ni pour la religion. Ce sont des
microbes cultivés: ils ont perdu leur virulence.

«Mais les nôtres sont certainement venimeux!» s'écrie un conservateur
furieux. Et il me rappelle les ignobles assassinats de Cullera, les
incendies, les cruautés de Barcelone, les pillages et les déprédations
commis ailleurs.

Il a raison. Pour l'instant, nos socialistes n'ont pas de chemises à se
mettre. Et manquer de chemise, cela ne vaut rien pour la moralité. «Il
n'est pas impossible qu'un pauvre soit honnête», disait Cervantés.
L'honnêteté est en effet une chose de prix et qui n'est généralement
qu'à la portée des personnes à leur aise. Le privilège le plus enviable
des riches, c'est de pouvoir se donner le luxe d'être honnêtes.

Il m'est cependant venu aux oreilles que quelques-uns des chefs du
socialisme espagnol ont maintenant des chemises de jour et de nuit, et
non seulement des chemises, mais aussi des maisons de rapport. On dit
même que ce sont d'impitoyables propriétaires et qui ne manquent jamais
le premier du mois, à l'heure du déjeuner, d'envoyer leur quittance aux
locataires, lesquels en perdent l'appétit et avalent de travers leurs
côtelettes pannées. Je ne crois pas à cette noire légende, elle a été
sans doute inventée et répandue par quelque réactionnaire malveillant.

En tout cas, nous devrions nous féliciter que les socialistes aient des
maisons de rapport. Et s'ils achètent des actions de la Banque
d'Espagne, ce sera mieux encore. Le jour où les socialistes espagnols
auront des jardins avec des grilles et conduiront leurs femmes à la
messe, les bourgeois n'auront plus à trembler pour leur titres de
propriété ni pour leurs tiroirs.

Dans tous les pays, les socialistes ont ajouté de nos jours à leur
bannière une devise séduisante: «A bas la guerre! Fraternité
universelle.» Et c'est vraiment très bien. Tout de suite j'ai été pris
par ce cri qui répond à l'aspiration la plus ardente de tout esprit
chrétien.

Fraternité universelle: le beau mot! Mais en attendant cette fraternité
si vaste, les bons socialistes ne pourraient-ils pas faire usage d'une
autre un peu moins étendue? Pourquoi voyons-nous tous les jours, quand
une grève se déclare dans quelque établissement industriel, que le
malheureux ouvrier qui se présente, poussé par la faim, pour reprendre
le travail, est assailli par ses compagnons avec une fraternité canine?

Il n'y a personne en Europe qui n'ait éprouvé quelque sympathie à voir
parmi les principes du socialisme moderne le désarmement des nations et
conséquemment la paix entre elles... On disait autrefois: «Paix entre
les princes chrétiens». Il aurait fallu ne pas supprimer cette phrase,
car ce sont les princes chrétiens qui ont été la principale cause de
cette guerre. Tous, jusqu'aux plus récalcitrants bourgeois, tournèrent
les regards vers eux avec une affectueuse complaisance. Dans les
ténèbres amoncelées sur la vieille Europe par des armements incessants
qui semaient l'épouvante dans les âmes, le seul rayon de lumière que
nous ayons perçu nous venait du socialisme. La diplomatie, nous
disions-nous, est impuissante: elle a perdu tout crédit; mais le
socialisme est fort, les masses ouvrières se chargeront d'opposer une
barrière à la superbe et aux ambitions des tyrans. Si elles laissent
tomber leur fusil et se croisent les bras, qui fera la guerre?

Nous avons été bien amèrement déçus. Les ouvriers ne laissèrent pas
tomber leur fusil. Tous s'empressèrent au contraire de l'empoigner et de
s'en servir avec une inconscience de soldats mercenaires.

Était-ce lâcheté! Était-ce l'effet de ce féroce instinct qui pousse les
troupeaux qu'on est parvenu à exciter? Je n'en sais rien; mais le fait
est vraiment lamentable. De toutes les faillites qu'a entraînées la
guerre, celle du socialisme est assurément la plus attristante. Causant
il y a quelques jours avec l'un de ses représentants, je lui exprimai,
non sans chaleur ni amertume, le sentiment de tristesse que ses
coreligionnaires avaient donné au monde dans cette guerre.

-- Est-ce la peine, lui disais-je, que pendant tant d'années vous ayez
prêché la paix et la fraternité internationale, fait systématiquement
obstacle aux armements, pour en arriver à être des guerriers aussi
féroces que les nôtres?

Et voici dans quels termes il répondit à mon interpellation:

«Pour tout le monde, socialistes ou bourgeois, des jours très durs se
sont levés. Quand dans une maison l'on crie «au feu!», les plus stoïques
sautent de leur lit; et si c'est «au voleur!» qu'on crie, le moins cruel
se saisira de son couteau de cuisine. Être pacifiste lorsqu'on a à côté
de soi un ennemi qui épie vos mouvements pour se jeter sur vous à la
moindre négligence, c'est un vrai crime. Eh bien, nous, les socialistes
français, nous l'avons commis, ce crime-là, et nous devons l'expier en
versant largement notre sang. Nous nous étions opposé aux dépenses
militaires; nous avons maltraité des généraux qui étaient braves et
prévoyants, pensant que nos frères de là-bas en feraient autant. Ils
faisaient bien quelque chose, mais nous voyons aujourd'hui que ce
n'était qu'une comédie, qu'au fond ils étaient les complices des tyrans
et que les uns et les autres s'entendaient pour s'élancer sur nous et
nous arracher le fruit de nos travaux. Toutes les lois, qu'elles soient
divines ou humaines, cèdent devant le droit de légitime défense. Ne vous
êtes-vous pas, vous, brillamment défendus à Saragosse et à Gérone quand
nous avons envahi votre territoire? Et vous saviez pourtant bien que
nous ne venions pas avec l'intention de nous saisir de votre bourse.
C'était bien différent d'aujourd'hui. Je reconnais que nous, les
Français, nous pénétrions injustement sur le territoire des autres. Ce
fut un mouvement de vanité exploité par un homme de génie. Auparavant
notre République avait elle-même été attaquée par ces autres. Mais nous
du moins nous avions en venant chez eux quelque chose à donner. Nous
apportions, en politique, les droits sacrés de l'homme, alors méconnus
ou foulés aux pieds en Europe; dans l'ordre civil, nous apportions un
Code que vous avez tous copié dans la suite. Nous avions remplacé un
régime despotique par un régime libéral, ou simplement un roi par un
autre. Après tout ils étaient Français tous les deux: l'un frère de
Bonaparte, l'autre petit-fils de Louis XIV. Et la preuve que nous
n'étions pas des bandits, c'est que vos hommes les plus éminents
d'alors, les Moratin, les Silvela, les Menendez Valdès, les Hermosilla,
d'autres encore, prirent notre parti. La même chose advint dans d'autres
pays. Le plus haut esprit que l'Allemagne ait eu jusqu'alors, Gœthe,
fut injurié dans sa propre patrie parce qu'il passait pour être notre
ami.

Mais l'Allemagne, qu'apporte-t-elle de neuf et de bon à l'Europe? Elle
n'a pas les poètes les mieux inspirés, ni les plus profonds philosophes;
ses lois ne sont pas les plus sages, ni ses mœurs les plus pures.
Elle a des hommes de science éminents. Il y en a d'aussi grands en
France, en Angleterre, en Italie et en Russie. Ce n'est pas à elle que
reviennent les plus étonnantes inventions modernes, mais aux pays de
Marconi et d'Edison. Au lieu de régime plus libéral et plus humain,
c'est l'autocratie militaire que les Allemands apportent. C'est eux qui
ont imposé à toute l'Europe cette servitude moderne qu'on appelle le
service militaire obligatoire. C'est eux qui se sont dressés contre la
généreuse entreprise du tzar Nicolas II se proposant le désarmement.
C'est eux qui ont fait échouer la Conférence de La Haye. C'est eux qui
entretenaient l'alarme dans le monde entier. En somme, que leur doit-on?
Un peu de chimie et beaucoup moins de sens moral.»

Je laisse à mon ardent interlocuteur la responsabilité de ces raisons,
qui, si elles sont excessives, sont cependant vraies dans le fond.

\end{chapter}


\begin{chapter}{Français et espagnols}

C'est, je crois, un sujet très délicat. Il faut y être
maître-équilibriste pour ne pas tomber dans de lamentables méprises.
Parler en ce moment des relations entre Français et Espagnols sans
blesser les uns ni les autres, c'est une entreprise dont les dangers
devraient me faire reculer. «Taisez-vous! méfiez-vous! Les oreilles
ennemies vous écoutent», disent partout à Paris des écriteaux. Je ne
suivrai pas ce conseil. J'ai, pour me risquer sur la corde tendue, un
balancier dont je me suis toujours servi avec bonheur. Ce balancier,
c'est la sincérité.

Mais l'écriteau en question se prête au commentaire. Tout d'abord il
montre que le caractère français est expansif. Les Berlinois n'ont sans
doute pas eu besoin de pareil avis. Et si mes compatriotes les Galiciens
étaient en guerre avec une autre puissance européenne (mais ils ne se
mettront jamais dans ce cas), ils n'en auraient pas besoin non plus.

J'avais un ami, précisément un Galicien, que je rencontrais dans la rue
après une longue séparation.

-- Quand donc êtes-vous arrivé? lui demandai-je.

-- Il y a trois jours, me répondit-il.

Mais il ajouta aussitôt, regrettant d'avoir laissé échappé la vérité:

-- Et un peu plus.

Il est évident que la France manque de maîtres comme celui-là.

Parlons donc sérieusement de notre amitié pour les Français.

Il va de soi qu'il y a des gens en Espagne qui n'aiment pas et qui
n'admirent pas la France. Vieux ressentiments, dépits, colères, voilà ce
qui monte à la surface dès qu'on remue un peu l'eau.

C'est l'histoire de tous les voisins. Quand on vit longtemps avec
quelqu'un dans un commerce étroit, les petits ennuis, les inattentions,
les injustices naturelles à l'égoïsme finissent par se déposer peu à peu
dans ce que les psychologues appellent la «subconscience». L'éducation,
le désir d'avoir la paix, la paresse concourent aussi à contenir tous
ces éléments de discorde. Mais il arrive un moment où quelque événement
imprévu leur ouvre la porte. Ils sortent alors avec fureur et brutalité,
l'œil injecté de sang.

Il faut convenir que jusqu'à présent les Français ne se sont guère
souciés de gagner notre sympathie. La presse en particulier n'a pas
hésité à nous tirer dessus et à nous manifester son mépris dans plus
d'une occasion. Quand le présent président de la République nous fit
l'honneur de venir nous voir, quelques-uns des journalistes qui
l'accompagnaient se sont montrés peu aimables envers nous. J'ai lu dans
l'une de leurs correspondances que les rues de Madrid étaient sombres.
C'est tout simplement ridicule: il y a dans d'autres capitales de
l'Europe des rues aussi sombres que celles de Madrid. Mais cela n'est
rien; un Français ne m'a-t-il pas dit un jour qu'il suffirait de $25000$
hommes pour conquérir l'Espagne!

Je sais bien qu'il y a partout des êtres grossiers et niais. Il ne faut
pas toutefois s'étonner que ces coups d'épingle aient fini par faire
l'effet d'un coup de couteau. Il y a peu de personnes capables d'assez
de sang-froid pour assigner aux choses leur valeur véritable. Un des
théorèmes de l'Éthique de Spinoza dit: «Celui qui croit être haï d'un
autre et ne lui avoir donné aucune raison de haine, haïra cet autre à
son tour.»

Tout cela, je le répète, c'est le voisinage. Si les habitants d'une
maison savaient comment ils parlent tout bas les uns des autres, cette
maison aurait vite fait de devenir un camp d'Agramant, et quand l'un
deux est assez sot pour le dire tout haut, c'est alors qu'éclatent ces
querelles de Capulets et Montaigus que nous connaissons tous.

Je crois d'ailleurs que si au lieu des Français nous avions les
Allemands pour voisins, les Allemands ne nous seraient pas plus
pitoyables. Témoin ce journaliste germain qui vint me voir il y a
quelques années. Notre nation le ravissait; tout l'intéressait, tout
l'émouvait; il courait tous les villages de la province de Madrid,
passait des semaines entières avec les paysans et apprenait d'eux de
grossières chansons, qu'il répétait d'une façon risible. J'avais
cependant quelques vagues soupçons que cette admiration pour l'Espagne
n'était pas de bon aloi. Et c'est lui-même qui vint un jour me les
confirmer.

-- Hier, dit-il, j'ai rencontré un de mes amis de Leipzig, un confrère,
qui est ici depuis quelques jours. Le malheureux se plaint de tout: de
vos chemins de fer, de vos hôtels, de vos services publics, de la poste,
du pavé de vos rues, de la police, de l'éclairage... J'ai fini par lui
dire: «Mon cher, tu es vraiment un peu sot. Ce n'est pas en Espagne
qu'on vient chercher de bons hôtels, ni des rues bien pavées, ni une
police, ni une poste... On vient chercher ici de tout autres choses!»

Je confesse que des couleurs de colère me montèrent au visage. Ce jeune
journaliste nous prenait pour des Africains et parlait de Madrid comme
il l'eût fait de Meknez.

En dehors de ces antipathies éparses, nées du dépit, il y a chez nous
des éléments puissants qui se sont mis du côté des Allemands dans le
présent litige. On peut dire sans crainte de se tromper que des trois
états, le clergé, l'armée et le peuple, le dernier seul a de la
sympathie pour les Alliés. Les deux premiers se sont rangés d'une façon
plus ou moins manifeste du côté des Empires centraux. Je vois bien sur
quoi se fonde le deuxième pour garder la position qu'il a prise.
L'Allemagne est un empire essentiellement militaire: il est normal que
tous ceux qui exercent en Europe le métier des armes aient quelque
inclination pour elle. Si l'on fabriquait en Allemagne plus de fruits au
sirop que d'explosifs et de liquides inflammables et s'il sortait des
usines Krupp, au lieu de canons, des gâteaux, tous les confiseurs
d'Espagne seraient germanophiles.

Quant à l'attitude du premier de ces états, elle me paraît moins
justifiée. D'où vient, de quoi procède l'amour que notre clergé régulier
et séculier témoigne pour les Allemands?

-- Ce n'est pas, me disait un ami, par amour des Allemands qu'ils sont
ainsi: c'est en haine des Français.

-- Impossible! répliquai-je. Dans la doctrine chrétienne, le mot haine
est vide de sens. Un ministre du Crucifié ne doit jamais agir que par
amour. Il est possible d'ailleurs de haïr une ou plusieurs personnes,
mais monstrueux et absurde de détester quarante millions d'êtres
humains.

Pour parler avec la sincérité promise, je dirai que je suis assez porté
à croire à l'existence de quelque révélation connue seulement des
religieux et des prêtres et cachée à la plupart de nous. Il est plus que
probable qu'une religieuse, dans quelque couvent d'Espagne, eut une de
ces visions célestes comme en ont eu sainte Thérèse ou son élève la
bienheureuse Marina de Escobar, et que Notre Seigneur, dans cette
vision, lui découvrit que nous devions nous mettre résolument du côté
des Germains et des Turcs. On a eu grand tort dans ce cas de ne pas
rendre publique la nouvelle de cette vision, car sa publication eût
permis aux fidèles chrétiens d'Espagne qui avons pris le parti des
Alliés de sortir de l'état de péché mortel où nous sommes.

Je comprends néanmoins que certains catholiques se soient laissés égarer
par cette loi d'association de sentiments, dont Spinoza a aussi parlé.
Lorsqu'une personne ou une chose nous a produit une impression
désagréable, tout ce qui se rapporte à cette personne ou à cette chose
nous produit le même effet. C'est ainsi qu'ils étendent à tous les
Français l'aversion que quelques-uns d'entre eux leur inspirent.

Le sectarisme en France avait fini par devenir odieux. C'était un
terrorisme blanc, à l'instar du terrorisme rouge de 93, dont le genre
humain garde encore le souvenir affreux. On n'y coupait point de têtes,
mais des carrières et des bourses. C'étaient des sacrifices non
sanglants, avec des conséquences désastreuses pour les victimes et leurs
familles. Comme au temps de Robespierre, le Pouvoir central avait ses
délateurs dans tous les coins de la République. Des renseignements sur
les fonctionnaires civils et sur les militaires arrivaient aux bureaux
des ministères de l'Intérieur et de la Guerre. C'était une Inquisition
renversée. Il y avait une liste de personnes qui se confessaient et
communiaient, une autre de celles qui n'assistaient qu'à la messe du
dimanche, une autre enfin de celles qui accompagnaient leurs femmes à
l'église et restaient à la porte. Est-ce assez ridicule? Il semble
impossible que les Français, si avisés d'ordinaire, si fins, d'un
sentiment du comique si aigu, aient pu supporter un ridicule de cette
taille-là.

Mais je ne vois pas qu'il y ait là de quoi les haïr. Ce n'est qu'une de
ces innombrables lâchetés sociales, comme on en observe dans tous les
temps et dans tous les pays. Un démagogue parvient à s'élever et sème la
terreur dans la nation, non plus comme ses anciens collègues au moyen de
la guillotine, mais par le retrait d'emploi et la disgrâce. Est-ce
étonnant? Qu'on se rappelle ces malheureux temps où notre Espagne était
dans les griffes d'une minorité anarchique et grossière. L'exercice du
culte catholique était alors soumis à des restrictions, on injuriait
dans la rue les ministres de ce culte, de répugnants blasphèmes étaient
proférés en plein Congrès des députés. Supposons qu'il ait alors existé
près de nous un peuple craignant Dieu et qui, sous le coup de ces excès
nous ait pris en mortelle haine et se soit réjoui de nos malheurs.
N'aurions-nous pas immédiatement crié à l'injustice? C'est précisément
la situation où se trouve aujourd'hui la France vis-à-vis de l'Espagne.

A tort ou à raison, une grande partie de cette France trouve que nous,
Espagnols, nous lui sommes hostiles. Les Français se sentent blessés et
s'irritent, et cette irritation se traduit en froideur, pour ne pas dire
plus. Quelques Espagnols, hommes et femmes, se plaignent à moi d'avoir
été reçus sans politesse dans certains lieux; que dans les magasins où
ils font leurs achats, ils ont entendu, prononcées à voix basse, de
désagréables paroles. «Mesdames, messieurs, leur ai-je répondu, ce qui
vous arrive là ne doit pas vous surprendre. On oublie aisément que
l'amour n'est pas aussi répandu qu'il conviendrait dans notre humanité.
Quand un chien étranger traverse un village, ceux du village lui aboient
tous sans raison. Entre gens qui se sont vus longtemps et qui semblaient
s'estimer, il suffit d'un rien pour amener la rupture et la haine. Qu'un
domestique nous insulte dans la rue et nous en voudrons à son maître qui
n'aura pas quitté son logis. Mon père avait un chien à qui il était
impossible de traverser certain quartier où nous passions quand nous
allions en promenade. Arrivé là, il devait s'en retourner, parce qu'il
avait dans ce quartier un frère de race qui lui était un ennemi
formidable. Un jour le maître de ce chien vint nous voir. A notre grande
surprise, notre chien qui était très pacifique se jeta furieusement sur
l'autre et ce fut une rude affaire que de l'empêcher de le mettre en
pièces. Tel est le monde des chiens; tel est aussi celui des hommes.
Nous payons à Paris les vitres que nos germanophiles brisent à Madrid.

Et pourtant je dois à la vérité de reconnaître que ni moi ni aucune des
personnes qui m'accompagnent n'avons rien entendu qui pût nous déplaire
dans notre voyage en France. Bien au contraire, on nous a partout reçus
avec la plus parfaite correction. Mes bons Espagnols ont sans doute été
victimes de leur imagination.

Mais en admettant même qu'il y ait dans le vulgaire quelque hostilité à
l'égard de la France, cela ne nous déconcerterait pas. Qu'est-ce que le
vulgaire? Ici et partout ailleurs il n'y a d'important que les gens qui
pensent, ceux que l'on s'est mis de nos jours à appeler les
«intellectuels». A Paris c'est quelques milliers de personnes; quelques
centaines à Madrid. Ceux-là ont de la stabilité dans les sentiments et
sont par conséquent dignes de respect. La masse penche d'un côté ou de
l'autre selon le vent; ce qu'elle aime aujourd'hui, elle l'aura demain
en horreur. La roche Tarpéienne a partout et toujours été près du
Capitole. Je me souviens qu'à mon premier voyage à Paris, il y a une
vingtaine d'années, on m'avait recommandé, si je voulais m'épargner des
ennuis, de faire tout mon possible pour n'être pas pris pour un Italien.
Il serait bon aujourd'hui de prendre en France l'accent napolitain ou
toscan.

Les intellectuels français sont avec nous. Ils ont reçu avec gratitude
le manifeste que leur adressèrent les nôtres. Ils savent apprécier nos
qualités et, pour dire toute la vérité, j'ajouterai qu'ils nous jugent
parfois meilleurs que nous sommes. Dans une étude sur la littérature
espagnole qu'a publiée naguère le savant professeur de la Sorbonne M.
Ernest Martinenche, je lis les lignes suivantes: «De toutes les
littératures étrangères, l'espagnole est peut-être celle qui a exercé en
France l'action la plus profonde et la plus continue.» Il est donc faux
que nous soyons en mépris aux seuls hommes capables d'apprécier. Et
comme en définitive c'est eux qui guident l'opinion et qui dirigent le
monde, nous ne pouvons qu'être sûrs de l'amitié de la France.

\end{chapter}


\begin{chapter}{Les femmes et la guerre}

Me promenant au Bois de Boulogne, voici quelques années, en compagnie
d'un Espagnol arrivé comme moi depuis peu à Paris, il nous arriva de
rencontrer un jeune et joli couple gracieusement embrassé. Il passa près
de nous le plus tranquillement du monde, sans paraître le moindrement
embarrassé d'être vu. Mon compagnon s'en montra profondément scandalisé:
il était arrivé tout disposé à se scandaliser.

A Madrid, la corruption parisienne est proverbiale. Tout est proverbial
à Madrid. Je veux dire que ce que l'un pense, l'autre aussi le pense, et
ainsi de suite.

Un de mes amis, très enclin au paradoxe, prétend qu'il y a deux cent
quarante personnes en Espagne qui pensent par elles-mêmes. Hormis ceux
qui ne pensent en aucune façon, et c'est la classe la plus nombreuse,
les autres pensent aux dépens du voisin.

C'est une plaisanterie qui n'est pas tout à fait dépourvue de vérité.
Nous autres Espagnols, qui avons été sur terre et sur mer de hardis
aventuriers, nous devenons, dès que nous nous lançons sur l'océan des
idées, de timides marins. Un voyageur américain assure qu'en Angleterre
on exige de chacun qu'il ose avoir une opinion propre, et qu'on pardonne
facilement à qui rompt avec les conventions pourvu que ce soit avec
esprit. On voit dans ce procédé une garantie de la force et du progrès
de la nation. Or, en Espagne, c'est justement le contraire qui se
produit. Ici, on voit d'un mauvais œil tout homme qui dit ou fait ce
que d'autres n'ont pas dit ou fait avant lui. On conte que l'Allemagne
est le pays de l'uniforme: l'Espagne l'est aussi; mais nous, c'est
intérieurement que nous le portons.

Pour en revenir à mon compagnon de promenade, je dois dire qu'il rugit
d'indignation.

-- Quelle honte! quel cynisme! Il faut venir à Paris pour voir cela!
s'écria-t-il.

-- Ce n'est pas la peine de faire un si long voyage, répondis-je. On voit
bien que vous ne fréquentez pas les allées du Retiro.

Paris, pour ce qui est des relations des deux sexes, n'est pas plus
corrompu que Londres, Berlin ou New-York. Songez qu'avant la guerre il y
avait à Paris une population flottante beaucoup plus nombreuse qu'en
aucune autre ville du monde. Tous les gais compagnons d'Europe et
d'Amérique s'y donnaient rendez-vous pour s'amuser.

Force est de confesser que la mauvaise renommée des Françaises leur
vient des Français mêmes. Ce sont leurs pères, leurs maris, leurs frères
qui les ont déshonorées aux yeux du monde; dans le théâtre et dans les
romans de ces cinquante dernières années, il n'est question que des
vilains tours que les femmes françaises jouent à leurs maris.
L'intempérance est à peu près la seule muse des romanciers modernes;
l'adultère leur seul sujet. De sorte que celui qui se sature de cette
littérature-là doit forcément penser qu'il n'y a en France ni femme
fidèle ni fille pudique, ce qui est une infâme calomnie.

Sortez de Paris et vous trouverez dans toutes les provinces de la France
les mêmes mœurs qu'en Espagne. Moi qui depuis longtemps passe une
partie de l'année dans une de ces provinces, je n'y ai jamais rien
observé de bien immoral. Assurément, il y a bien çà et là quelques
divorces; mais les dames françaises regardent de travers la femme
divorcée, tout comme cela se ferait en Espagne. D'ailleurs, nos lois y
consentant, n'y aurait-il pas de divorces chez nous!

Et puis, la Française a tant de choses à faire valoir, qu'on peut bien
lui passer un peu de coquetterie. Elle a pour elle sa grâce, son
intelligence, son élégance, sa culture; elle a surtout l'inlassable
besoin de se rendre aimable. Ce n'est pas dans les hommes, mais dans les
femmes, que réside la fameuse courtoisie française. J'en demande bien
pardon à tous mes bons amis de France.

Le pouvoir de la femme française est infini. Personne ne lui résiste.
Parfois sans beauté, souvent sans haute position sociale, sans riches
habits, ni instruction solide, elle sait cependant fasciner, puis
s'assujettir ceux qui l'approchent. On est étonné, lorsqu'on lit la
correspondance de Voltaire, de l'immense variété de phrases ingénieuses
dont disposait cet homme pour flatter ses correspondants. Or, toutes les
Françaises sont de petits Voltaires. Quand en France vous entrez dans un
cercle de dames, soyez sûr que vous y entendrez maintes petites phrases
flatteuses pour votre amour-propre et dites avec un tel art, une
simplicité si raffinée, que vous ne vous rendrez pas compte qu'on vous
adule. Et cela constitue un vrai péril: vous vous retirerez en faisant
la roue comme un paon.

Il est remarquable qu'à mesure qu'elle vieillit la Française devient
plus aimable. Si les Anglaises, comme le disent les romanciers et les
voyageurs, aigrissent avec le temps, les Françaises sont comme les
confitures: elles concentrent leur douceur et se givrent en
vieillissant. C'est alors qu'elles déploient tous les recours de leur
art. Il est difficile en France de se défendre d'une jeune femme; mais
résister à une vieille, impossible.

Il y a quelques jours, j'attendais le tram à une station. Je ne savais
pas qu'il fallait arracher d'une certaine colonne un petit papier avec
un numéro. Une dame aux cheveux gris s'aperçut de mon involontaire
insouciance.

-- Monsieur, me dit-elle, vous feriez bien d'aller chercher un numéro,
sans quoi vous ne prendrez jamais le tram.

Une autre fois, dans une église, j'oublie mon manteau sur le prie-dieu
où je m'étais agenouillé. Je me trouvais déjà à la porte, quand je sens
derrière moi une respiration haletante et j'entends une voix qui me
disait:

-- Monsieur, votre pardessus que vous aviez oublié!

C'était encore une dame avec des cheveux blancs. Comment ne pas adorer
ces bonnes vieilles françaises?

Autre particularité curieuse: en France, contrairement à ce qu'on
observe en Espagne, il n'y a pas de provinciales. Toutes les femmes sont
parisiennes. Même goût dans le vêtement, même esprit, même politesse,
même distinction dans les manières. Dans un village, en plein air, j'ai
vu d'humbles paysannes danser avec une élégance et une majesté telles
que si une fée eût soudain changé en soie le percale de leurs habits et
en orchestre le misérable violon qui accompagnait leurs pas, on se fût
cru au milieu de princesses. Tout en nous promenant, nous entendions des
personnes qui se saluaient en termes cérémonieux et entamaient une
conversation où s'échangeaient de fines idées. Nous tournons la tête: ce
sont des domestiques qui ont rencontré un employé de tramway. J'ai même
été témoin d'une discussion entre deux femmes, qui en vinrent aux mains
sans abandonner cependant toute courtoisie.

-- Oh, madame! criait l'une en lançant un coup de griffe à l'autre.

-- Oh, mademoiselle! faisait l'autre, la main en l'air pour la saisir aux
cheveux.

Quant à la politique, si presque tous les hommes en France sont
républicains, il est rare qu'une femme le soit. Du moins, toutes les
femmes que j'ai rencontrées m'ont interrogé sur notre roi, sur la reine,
sur les princes et les infants, avec un intérêt, une sympathie qui
révèlent des sentiments monarchiques encore tièdes. Elles manifestent la
plus vive curiosité pour les particularités de la vie et pour les
habitudes de notre famille royale. J'avais beau leur dire que n'étant
pas courtisan et n'allant jamais au palais, il m'était impossible de
leur donner satisfaction, elles s'obstinaient, voulaient tirer de moi
quelque détail amusant, une nouvelle, une anecdote. Alors, me souvenant
que j'étais romancier, je leur contai une histoire.

Leur attitude, la guerre déclarée, fut absolument admirable. Je les ai
vues pleines de confiance, sereines, résolues comme les hommes, mais
avec plus de dignité encore. Devant moi, quelques-uns de ceux-ci,
complètement affolés, se laissèrent aller à injurier l'ennemi, à
proférer contre lui des paroles de mauvais goût. Jamais les femmes ne
s'abaissaient à l'injure grossière. Elles, si communicatives
d'ordinaire, restaient graves et silencieuses. Mais dans leurs yeux,
dans toute leur personne, on lisait l'inébranlable décision d'aider
leurs maris, leurs frères jusqu'à la mort.

Et ce qu'elles l'ont accomplie, cette décision! Dans une guerre
d'agression et de conquête, la femme est peureuse. Pour marcher il faut
qu'elle se sente accompagnée de la justice. Mais quand elle la sent à
son côté, elle est alors plus intrépide que l'homme. Souvenez-vous,
Espagnols, des remparts de Gérone défendus par nos héroïques aïeules:
«Pas de quartiers! criaient-elles! Nous n'en faisons ni n'en voulons.»

Une fois convaincues que leur patrie avait été injustement attaquée, les
Françaises, pour alléger le sort des leurs, déployèrent les merveilleux
recours de leur propre nature. Aux champs, elles prirent sur leurs
épaules la lourde charge des cultures; ici, à Paris, elles remplissent
avec un égal succès les emplois des hommes. Et cela n'est pas sans
inquiéter ces derniers. C'est ainsi qu'un ouvrier me disait il y a
quelque temps, sur un ton d'amertume:

-- Voyez, monsieur; les femmes ont déjà tout envahi: elles sont
encaisseurs de tramways, garçons de café, employés de commerce, cochers,
elles travaillent dans les usines et même aux munitions. Qu'est-ce qui
se passera après la guerre? Les hommes trouveront toutes les places
prises et ils auront bien de la peine à les reprendre. La femme se
contente d'un salaire moitié moindre que celui d'un homme. Il va de soi
que les entrepreneurs et les propriétaires d'établissements commerciaux
préféreront conserver les femmes. Un grave conflit en sortira, vous
pouvez me croire.

Oui, je le crois. Mais je n'ai pu m'empêcher de me demander: quelle est
la cause originale de ce conflit? Ce sont les principaux besoins des
hommes, et pour parler très nettement, nous pourrions dire: leurs vices.
La femme n'a pas besoin d'alcool ni de tabac; elle est plus sobre dans
sa nourriture; elle n'exige pas des plaisirs coûteux. Il n'y a qu'une
façon de résoudre le problème: c'est que les hommes deviennent plus
sobres, plus soumis à leurs devoirs et se résignent à vivre avec le même
salaire que les femmes. Ils y gagneraient, et leur nation, leur race
tout entière y gagneraient aussi.

Des milliers de jeunes femmes dans une situation brillante, abandonnant
les commodités du foyer, allèrent servir dans les ambulances du front;
d'autres entrèrent dans les hôpitaux, dont quelques-uns se trouvent dans
les lieux les plus retirés du territoire, pour y recevoir les blessés;
d'autres enfin courent le pays, faisant tout ce qui est possible
humainement pour trouver des secours.

J'ai été témoin de leurs travaux dans ces hôpitaux. Elles ne se bornent
pas à entourer de soins les blessés, à les veiller, à nettoyer leurs
plaies: elles font beaucoup plus. Comme elles savent que la gaieté est
le plus efficace des médicaments connus, et capable à lui seul de
merveilleuses cures, elles s'efforcent de donner de cette gaieté à leurs
malades. La première chose qu'elles font pour cela, c'est d'installer
un piano, et si possible, un cinématographe. Alors, selon les
circonstances et l'état des blessés, elles organisent des concerts
vocaux ou instrumentaux, jouent des comédies, lisent des romans, font
des tours de prestidigitation et surtout rient, bavardent, charment les
malades.

Inutile d'ajouter que le petit dieu ailé, fils de Mars et de Vénus,
accourt dans ces lieux qui devraient être l'abri de la douleur et qui
sont souvent celui de l'allégresse. Avec une cruauté inouïe, il achève
l'œuvre des Allemands en tirant sur ces malheureux, non plus comme
aux temps antiques des flèches d'or, mais d'ardentes grenades à mains.
Quelques-uns d'entre eux vont se rétablir à la sacristie de la paroisse;
d'autres repartent pour le front. Mais ceux-là promettent à leurs
infirmières qu'ils leur reviendront bientôt, à nouveau blessés.

\end{chapter}


\begin{chapter}{Auteurs et livres}

Après les hommes politiques, nous les hommes de lettres, nous sommes ce
qu'il y a de pire en tous pays. La politique est le domaine de l'intérêt
et de la vanité. Un artiste se passera sans peine de déjeuner si vous
daignez lui louer ses œuvres; et si vous lui dites du mal de celles
de ses confrères, peut-être se passera-t-il en outre de dîner. Mais, en
plus de l'éloge, il faut à l'homme politique du champagne et de bons
cigares. Toutefois, en ce qui concerne la flatterie, il a le palais
moins fin que l'écrivain. Quand j'étais jeune et que je fréquentais des
politiciens, j'en ai vu qui avalaient avec délectation de vrais plats
de gargote.

Cependant, il est trop souvent question de la vanité des poètes, comme
si ceux qui ne sont point poètes étaient exempts de toute vanité. Dans
ce monde-ci, tous ceux qui ont fait une œuvre, et même aussi ceux qui
n'en ont jamais fait, tous se jugent dignes d'être célébrés.

On prétend que de tous les grands écrivains c'est le Français qui est le
plus chatouilleux, le plus impatient. Je ne sais pas si c'est vrai,
n'étant en relations personnelles avec aucun. Tout ce que je sais, c'est
qu'en Espagne un de ses jeunes admirateurs ayant un jour demandé à un
poète fameux quel était de Shakespeare ou de lui le plus grand poète:

-- Je te le dirai, répondit gravement le poète espagnol, décidé à
éclaircir l'affaire.

Je ne crois pas que Victor Hugo fût allé plus loin.

Quoiqu'il en soit, je pardonne leur impertinence aux écrivains. Et si le
lecteur veut bien aussi la leur pardonner, il n'a qu'à faire comme moi:
c'est de vivre loin d'eux.

Un de mes amis, grand amateur de toros, me disait: «Les courses me
ravissent; mais je déteste les toreros. Si j'étais un despote à la
Caligula, la fête finie, je les ferais jeter en prison et ils n'en
sortiraient qu'à la course suivante.» Faisons-en autant; enfermons les
auteurs dans la prison de leurs livres et ne les en tirons qu'aux
moments où nous avons besoin d'eux. Je me trouvais à Paris, alors que
Zola, Daudet, Maupassant, Renan et Taine étaient du monde des vivants.
Malgré la grande admiration qu'ils m'inspiraient, je ne fis aucun pas
pour entrer en relation avec eux. En revanche j'en fis beaucoup pour
visiter les tombes de Balzac et de Musset. Et je peux assurer qu'ils me
reçurent d'une façon tout à fait cordiale et que je n'eus pas à me
plaindre de leur orgueil\footnote{Cet article avait paru dans l'Imparcial,
lorsque j'eus l'occasion de faire connaissance avec quelques
écrivains français éminents. Ils m'ont traité avec plus de courtoisie et
d'amabilité encore que Musset et Balzac. Tout ce que je viens d'en dire est
donc à effacer.}.

D'ailleurs il n'y a pas à s'étonner que les artistes et les écrivains
français se disputent avec acharnement les rayons de soleil de la
gloire. C'est qu'en France la gloire existe vraiment. Les artistes et
les écrivains y composent la plus haute aristocratie sociale, et le
public, sans qu'ils soient précédés de licteurs ni de faisceaux, leur
fait la haie et les salue avec respect. Mais en Espagne cette gloire
n'existe pas, n'a jamais existé. J'espère cependant qu'elle finira par
exister, car il ne faut pas que nous continuions à être jusqu'à la fin
des temps le peuple le plus rustre de l'Europe. Quand je songe à ces
malheureux et faméliques écrivains de notre dix-huitième siècle, qui
passèrent toute leur vie à s'injurier au milieu de la plus parfaite
indifférence du public, je suis pris tout ensemble de l'envie de rire et
de pleurer.

En France les écrivains ne se disputent pas que la gloire, ils se
disputent aussi l'argent. Car la littérature rapporte de l'argent, mais
assurément beaucoup moins qu'on ne le dit en Espagne. Du reste, les
gains de ces auteurs ne sont pas comparables à ceux de leurs confrères
d'Angleterre ou des États-Unis. Cependant leurs gains sont
considérables, mais leur gloire plus considérable encore. Aussi
lutte-t-on en France avec rage et fait-on d'incroyables efforts pour
l'acquérir. Ces efforts atteignent parfois même le comble du ridicule.

Ce qui explique cette soif de gloire, c'est qu'en France la littérature
tient une place énorme dans la vie. Tout le monde lit, les petites gens
comme les gens du monde, les hommes aussi bien que les femmes. Le nombre
des librairies est surprenant. Dans l'une d'elles, j'ai dû faire queue
pour acheter un livre. La demoiselle qui vous vend des gâteaux ou des
cravates vous parlera des dernières publications littéraires avec une
sagacité remarquable et parfois même de notre littérature avec plus
d'expérience que certains millionnaires espagnols. Après la guerre,
appauvris, épuisés par le malheur, les Français trouveront toujours de
quoi s'acheter des livres. Tandis que la maison Nelson a dû s'arrêter de
publier des ouvrages espagnols, elle continue de mettre tous les mois en
vente quelques volumes en français. Et pourtant, jusqu'aujourd'hui du
moins, nous n'avons eu aucune charge extraordinaire à supporter.

C'est pourquoi, habitués à être excessivement choyés et fêtés, à être
connus du monde entier, à voir leurs propos, leurs gestes, et jusqu'à
leurs éternuements, reproduits dans les lieux les plus reculés, c'est
pourquoi de temps en temps les écrivains français prennent une voix
grave et laissent échapper des sottises. Il faut avouer que la guerre
leur a fourni l'occasion d'en proférer pas mal.

Dans un roman de Balzac, un aristocrate français qui rentre chez lui
après les guerres de Vendée, transi de corps et d'âme par l'égoïsme de
quelques-uns de ses compagnons, se contente de dire avec une magnanime
simplicité: «Les barons n'ont pas tous fait leur devoir.» De même
pouvons-nous dire aujourd'hui: «Tous les écrivains n'ont pas gardé leur
dignité.» Ils ont écrit et publié de nombreuses fanfaronnades ridicules,
des menaces, des phrases inconvenantes. Et le pire, c'est que tout cela
se disait sans émotion et seulement pour attirer l'attention du public.
Voilà la plaie de la littérature française. Les écrivains perdent de
leur initiative et de leur liberté sacrée pour se faire les laquais de
l'opinion. Nous, leurs confrères d'Espagne, nous avons sur eux à ce
point de vue un avantage enviable. Que nous écrivions droit ou tordu,
comme un ange ou comme le diable, nous savons d'avance que le grand
public ne se soucie point de nous. Nous travaillons pour une douzaine
d'amateurs; nous sommes libres comme l'oiseau de Minerve.

Oh, liberté sacrée, nous ne paierons jamais assez tes caresses! J'ai
toujours senti tes baisers sur mon front quand je traçais les humbles
ouvrages que j'ai livrés au public. Mais j'assure qu'ils ne m'ont jamais
été si doux, ces baisers, qu'aux jours où, tout jeune homme, je
descendais l'escalier d'un politique éminent chez qui je venais de
passer quelques heures. «Dieu! m'écriai-je, les yeux au ciel. A quoi
sert d'avoir la gloire et le pouvoir si l'on est obligé d'écouter de
pareilles inepties? Infortuné grand homme! Modeste gribouilleur de
papier, du moins ne suis-je pas comme toi l'esclave des grandeurs. Je
suis libre. Je peux à l'instant même aller m'asseoir sur un banc de
Recoletos ou manger un bifteck au café Habanero: la foule de tes
flatteurs ne m'y suivra pas.»

Les écrivains français prêtent trop l'oreille aux rumeurs de la rue.
Comme les rois, ils essaient leurs saluts au miroir; ils ne peuvent,
comme les enfants, se passer de cajoleries. Ils auraient besoin d'une
école plus rude pour acquérir un peu de simplicité. Reconnaissons
toutefois que le bon sens, cette pudeur de l'esprit gaulois, s'imposa à
eux après les premiers jours de la guerre. Il y a beau temps aujourd'hui
que les phrases de mauvais goût ont disparu des journaux. On n'y écrit
à présent qu'avec mesure et dignité.

\horizontalLine

Je me trouvais ces jours-ci à Montmartre, sur la terrasse du
Sacré-Cœur. C'était à la tombée du jour, heure de mélancolie. Le
panorama que découvraient mes yeux est unique au monde. La grande Lutèce
étendait le toit de ses maisons jusqu'aux derniers confins de l'horizon.
Le soleil, tour à tour caché dans les nuages et soudain reparu, jouait
avec la ville qui s'éclairait ou s'assombrissait à son gré. Là, une
brume bleuâtre donnait un sentiment de paix idyllique; ici, un nuage
noir inspirait la tristesse et la crainte. Le Trocadéro, la tour Eiffel,
les Invalides, le Panthéon, Saint-Sulpice, Sainte-Clotilde, Notre-Dame
rappelaient à mon esprit les faits les plus saillants de l'histoire
ancienne et moderne.

Jamais je n'ai senti comme à cette minute l'importance de la grande
cité. Victor Hugo disait de Paris que c'était le cerveau du monde. Ce
n'est là qu'une de ces phrases sonores comme en a proféré beaucoup ce
génie emphatique. Non, Paris n'est pas le cerveau du monde: il y a bien
d'autres endroits où l'on pense; il y a partout des cerveaux. Paris,
c'est la main du monde. Nous vivons tellement séparés les uns des autres
sur cette planète, et non seulement par la distance physique mais encore
par la distance morale, ce qui est pire, que s'il n'y avait pas une main
pour nous conduire les uns vers les autres, nous courrions le danger de
nous glacer dans notre solitude.

Grande et noble destinée que celle de la France! Tous nous venons nous y
laver de notre exclusivisme. C'est le centre où s'équilibrent toutes
les forces; c'est l'alambic où se distillent tous les mauvais goûts et
toutes les grossièretés dont le monde est entaché. La France est comme
un grand salon et Paris une maîtresse de maison qui sait avec un tact
raffiné faire observer une attitude correcte à ses hôtes les plus mal
élevés.

Si les Allemands avaient vaincu la France, ils eussent tôt ou tard été
soumis au joug aimable de cette ravissante Circé, comme autrefois les
Romains à celui d'Athènes.

La France se charge de faire la balance des grandeurs et des petitesses
des hommes. Quand ils arrivent à Paris, les rois les plus despotiques se
transforment en citoyens aimables et les humbles ouvriers en hommes de
bonne compagnie. Tout le monde ici se fait la barbe et ôte ses bottes de
cheval. Les Peaux-Rouges d'Amérique vous y demanderont pardon de passer
devant vous.

On me dira que tout cela n'est que l'apparence, et que l'important est
d'avoir l'intelligence élevée et le cœur droit. D'accord; mais la
courtoisie est un antidote contre l'égoïsme et le commencement de la
charité. On arrive aux sentiments par les actes, disent les psychologues
modernes; Pascal prenait de l'eau bénite pour se donner la foi. La
nature humaine est si vicieuse qu'il lui faut tous les freins de
l'éducation pour qu'elle ne montre point sa lèpre.

Mais elle n'est pas que distinguée et charmante, cette maîtresse de
maison: elle est en outre plus cultivée qu'aucune. L'Angleterre a une
littérature plus riche, l'Allemagne une philosophie plus haute, l'Italie
un art plus splendide. Pourtant, considérée dans l'ensemble, c'est la
France qui l'emporte. Sa littérature du dix-septième siècle est
admirable. Les noms de Corneille, Racine, Bossuet, Fénelon, Mme de
Sévigné, Molière, La Rochefoucauld rivalisent avec les noms les plus
grands des autres pays. Au dix-huitième, il y a des colosses comme
Voltaire, Diderot, Rousseau, et d'exquis écrivains comme Marivaux,
l'abbé Prévost, Beaumarchais et Champfort. Le dix-neuvième est
merveilleux: Chateaubriand, Lamartine, Hugo, Musset, Vigny, Balzac,
George Sand, Michelet ont vécu dans le même moment, et à côté d'eux des
douzaines d'écrivains notables, tels qu'aucune nation n'en saurait
montrer.

Et si nous en venons aux sciences, c'est mieux encore. L'Allemagne est
la première dans l'application industrielle; mais dans la science pure
les Français ont été et continuent à être les maîtres. Descartes,
Malebranche, Pascal, Laplace, Lavoisier, Lamarck, Champollion, Ampère,
Gay-Lussac, Buffon, Cuvier en sont la preuve; et de nos jours Pasteur,
Auguste Comte, Claude Bernard, Quatrefages, Charcot, Taine,
Brown-Séquard. Dans ces dernières années il n'y a pas eu de naturaliste
comparable à Pasteur, ni de mathématicien comparable à Henri Poincaré,
mort naguère, ni de métaphysicien égal à Bergson, gloire de son pays. En
ce moment Le Dantec, Boutroux, Pierre Janet, Grasset, Richet, Durkheim,
Le Bon, et tant d'autres qu'il m'est impossible de nommer, travaillent
avec éclat.

Quand je me rappelle tant de noms illustres, quand j'observe cette
jeunesse si avide de s'instruire et que je considère le travail efficace
et harmonieux que font en même temps ici les savants naturalistes et les
penseurs, les prêtres et les militaires, les ouvriers et les écrivains,
je ne peux m'empêcher de tourner les regards vers cette Espagne que
j'aime tant. Mon cœur se serre et un flot d'amertume me monte à la
gorge et m'étouffe.

Ce peuple espagnol, je me le représente comme un homme bien doué, bien
bâti, d'une intelligence pénétrante, mais endormi. Je voudrais qu'un
génie puissant, un nouvel Ariel, parût et le secouât rudement en lui
criant dans l'oreille: «Éveille-toi! Éveille-toi! N'entends-tu pas le
chant de l'alouette? Ne vois-tu pas que le soleil crible déjà la terre
de ses rayons? L'œuvre est longue. Presse-toi! L'humanité attend
beaucoup encore du pays qui lui a donné Cervantés et découvert de
nouveaux mondes. Dans la marche du progrès, qui n'avance pas recule. Si
tu continues à dormir, la poussière finira par faire croûte sur toi; les
araignées et les rats te grimperont dessus et les moutons imprimeront
leur ongle sur ton visage.»

Peut-être l'endormi s'éveillera-t-il; peut-être alors se frottera-t-il
les yeux et après un instant d'hésitation répondra-t-il: «Pourquoi?»
Puis, se tournant de l'autre côté, il se remettra à dormir.

Mais peut-être aura-t-il raison. Une fois debout, que verrait-il? Des
campagnes desséchées, des hommes affamés, le népotisme dictant ses
ordres, l'injustice dressée en système, la frivolité lâchant des éclats
de rire stupides, une politique mesquine empoisonnant les plus hautes
intelligences et les caractères les plus nobles...

Dors donc, peuple espagnol, dors! Il vaut mieux vivre endormi, qu'être
éveillé mais sans espoir.

\end{chapter}


\begin{chapter}{Le krischna des tranchées}

La répétition est la loi de la vie. Les faits se répètent et les pensées
aussi. Ce qu'ont pensé nos plus lointains ancêtres quand ils
commencèrent à penser, nous le pensons nous-mêmes aujourd'hui.

Devant la nécessité inéluctable, l'homme, poursuivi par les rigueurs de
la nature, se réfugie dans sa propre âme et adopte un stoïcisme
fataliste qui l'émancipe de la douleur. Toute la philosophie de l'Orient
est imprégnée d'un pareil stoïcisme; la philosophie grecque acquit le
sien au Portique; les plus grands hommes de l'antiquité lui rendirent
un culte. Et de nos jours mêmes, quand la foi chrétienne n'adoucit point
notre amertume, chacun de nous lutte contre la douleur en mettant son
âme pointe en avant contre les événements et en livrant sa pensée à
l'oracle de la fatalité.

De tous les oracles fatalistes, celui qui s'exprime dans l'épisode du
Mahabharata indien connu sous le nom de Bhagavad-Gita est le plus fameux
et le plus impressionnant. Les armées des Pandavas et des Curavas se
trouvent en face l'une de l'autre dans une plaine immense. Les cornes de
guerre sonnent, les tambours battent, les chars se précipitent, les
flèches sifflent. Krischna, incarnation humaine de Vichnou, consent à
servir de cocher au troisième fils de Pandou, Ardjouna, son disciple
favori. A la vue de tous ces hommes qui vont s'égorger, Ardjouna se sent
pris d'une mélancolie désespérée. Il contemple cette multitude d'amis
et d'ennemis que sépare la haine et que la mort va unir, et ses mains
tremblent, sa bouche se sèche, ses cheveux se dressent, la peau lui
brûle, ses forces tombent, son arc lui échappe des mains. Il se laisse
défaillir sur le siège de son char, pâle, effrayé, l'âme transie de
douleur. C'est alors que Krischna lui révèle qui il est, et commence à
l'instruire de la vanité des choses terrestres et du caractère
insignifiant de tous nos actes. Le vrai sage ne doit s'inquiéter ni des
vivants ni des morts: le corps n'est que l'enveloppe d'une intelligence
immortelle, qui change de forme comme un habit. Mourir ou tuer, c'est
absolument indifférent, etc., etc.

Là-bas, dans les tranchées de la Champagne, cette même scène s'est
répétée. Il ne s'agissait plus de dieux, mais de pauvres soldats de
l'infanterie. Voici comment j'en ai eu la nouvelle.

Je venais d'entrer avec un ami dans un café des boulevards. Au moment de
nous asseoir, mon ami aperçut au fond de la salle quelqu'un qu'il
connaissait et il s'empressa d'aller le saluer. Je vis que son
interlocuteur avait deux béquilles près de lui et je pensais
immédiatement que c'était un invalide de la guerre. Mon ami me fit signe
alors de m'approcher; il me présenta à l'invalide et nous prîmes place à
sa table. C'était un garçon qui avait l'aspect agréable, l'air ouvert et
bon. On lui avait coupé une jambe, il n'y avait pas longtemps; c'était
le fils d'un banquier du boulevard Haussmann et il paraissait jouir
d'une brillante situation dans le monde.

Il va de soi que la conversation roula sur la guerre. M. Gardiel-- ce
sympathique jeune homme se nommait ainsi-- nous entretint longtemps de la
vie des tranchées; il nous conta quelques-unes de ses aventures
guerrières. Son récit n'avait rien d'extraordinaire; les journaux en
ont publié mille semblables. Je l'écoutais cependant avec attention: le
banal devient intéressant lorsqu'il est rapporté naïvement et par la
personne même qui l'a vécu. Un des épisodes de cette histoire commune
sortit tout à coup de l'ordinaire et me toucha profondément. Le voici en
quelques mots.

-- Parmi les hommes de la compagnie à laquelle j'appartenais, dit-il, il
y en avait un que sa laideur distinguait du reste. La nature en lui
semblait s'être surpassée. Je crois bien que c'était l'homme le plus
laid de France. Entre nous, nous l'appelions «la Mérode», en souvenir
d'une beauté qui a fait grand bruit naguère. Et le moral dans ce garçon
répondait assez bien au physique. Taciturne, brusque, indifférent à ce
qui se passait autour de lui, il nous était antipathique à tous. Ce
qu'il avait de plus repoussant, c'était le sourire: un sourire
sardonique, malicieux, qui ne lui tombait jamais des lèvres. Si une
grenade l'avait mis en morceaux, nous ne l'aurions pas regretté.

De son vrai nom il s'appelait Tabourin; on m'a dit qu'il était
professeur dans un collège de Lyon. Sa vocation scientifique était
patente: il profitait de toutes les occasions qui s'offraient à lui pour
faire la chasse aux insectes, aux papillons. Il les fixait ensuite sur
de petits bouts de carton qu'il gardait soigneusement dans son havresac,
ce qui nous le rendait encore plus antipathique. Son indifférence
glaciale était répugnante. Quand il entendait qu'on se plaignait de
l'humidité, de la faim, d'un mal quelconque, ses yeux avaient un regard
plus sarcastique. Lui ne proférait jamais une plainte.

«La grande offensive de septembre arriva. Les horreurs d'un enfer
imaginé par un dévot hystérique ne seraient rien à côté de celles que
nous avons connues pendant quelques jours. Nous avons vu couler tant de
sang, tant de membres s'éparpiller, nous avons entendu de tels cris de
douleur que, pour ma part, j'avais fini par être dans un état de stupeur
indicible.

«Une nuit, j'étais couché dans le fond de la tranchée, fatigué à
m'évanouir, mais incapable de dormir. J'entendais respirer mes
camarades; je songeais à ce que nous apporterait le matin prochain,
peut-être cette nuit même; je pensais à nos familles, à nos mères, et
j'étais triste jusqu'à la mort. Je ne pleurais pas: à la guerre on perd
la faculté de pleurer; mais je ne pouvais me retenir de soupirer.

-- Tu ne peux pas dormir, hein? me murmura quelqu'un à l'oreille.

C'était Tabourin.

-- Non, fis-je sèchement.

-- Tu es triste?

-- Oui, répondis-je sur le même ton.

-- Veux-tu de l'éther? J'en ai encore un peu.

La douceur de sa voix me surprit: c'était un tel contraste avec son air
repoussant! Je n'acceptai pas son offre; mais ému de reconnaissance, je
lui dis:

-- Non, je ne suis pas triste, du moins de ce qui peut m'arriver demain.
Être tué d'un coup de fusil ou de baïonnette, c'est peut-être ce qu'il y
a de mieux. Ce qui m'attriste, c'est de voir dormir tranquillement tous
ces pauvres diables et de penser à tout ce qu'il leur reste à souffrir.
C'est aussi de penser à tous ceux qui les aiment, et qui les pleurent et
les pleureront.

Il demeura quelques instants silencieux; puis, approchant ses lèvres de
mon oreille, il dit doucement:

-- Le sang, ce n'est rien; les larmes, moins encore. Qu'importe de
mourir! Je crois que ce doit être un plaisir immense que de se reposer
dans le sein de la grande Nature. Avec quel calme on doit dormir sous
quelques pelletées de terre! En réalité, mon cher, la mort n'existe pas;
l'étincelle de vie qui nous anime ne s'éteint pas avec chacun de nous:
elle va allumer un autre foyer. Les champs, les mers, les hommes, les
bêtes, les soleils qui brillent dans le ciel, tout ce qui se meut et
respire, tout naît et tout meurt, tout tombe et tout renaît. Seul le
grand pouvoir de la Nature ne s'éteint jamais, il est seul immortel. Ce
grand pouvoir silencieux et tranquille est le seul qui existe vraiment:
nous ne sommes, nous, que des apparences, que les projections du grand
cinématographe. Pourquoi la destruction nous ferait-elle horreur? Elle
aussi n'est qu'apparente. Vois les fourmis: elles traversent la route en
file, accomplissant leur tâche. Le pied d'un passant en écrase une
centaine; les autres poursuivent impassiblement leur ouvrage sans
donner d'importance à l'accident. Pourquoi en donnons-nous tant, nous
autres, à la mort d'une centaine d'entre nous? Nous et elles, nous
tombons également dans le sein fécond de notre mère la terre. Jamais le
Destin ne pourra nous priver de ce giron maternel. Le secret de la force
des choses est en nous comme en tout le reste des êtres. Il n'y a pas de
vide dans l'Univers. Les limites entre le monde inanimé et le monde de
la vie sont imaginaires... Console-toi, mon ami; la mort n'est une porte
d'horreur et de ténèbres pour personne: c'est au contraire le passage
d'une heure sombre à une heure claire. Soumettons-nous gaiement à la
volonté de la nature et ne voyons pas en elle une ennemie, mais une
tendre alliée qui nous délivre de l'intolérable tyrannie de la vie.

Naturellement, cela ne me consolait pas; mais dès lors j'eus du respect
pour ce compagnon qui était tout autre que ce que mes camarades et
moi-même nous nous étions figuré.

La grande offensive se termina; notre compagnie avait à peu près perdu
la moitié de ses hommes; je m'en étais tiré par miracle, et Tabourin
aussi. Nous revînmes à la vie monotone et malpropre des tranchées. Tous
ceux qui l'ont soufferte se la rappelleront avec dégoût. J'essayai
d'avoir avec Tabourin des rapports plus étroits; car après les graves
paroles que j'avais entendues de lui, il me semblait que son âme n'était
pas sans noblesse. Seulement mes attentions se brisèrent contre son
attitude toujours ironique et glaciale. Il continuait à nous fuir; il
parlait très peu et sur un ton presque toujours méprisant. Il devenait
chaque jour plus antipathique à ses camarades et plus odieux à ses
chefs.

Tabourin passait de nouveau ses loisirs à la chasse des lépidoptères,
dont il étudiait les antennes et les trompes et les écailles des ailes à
travers une loupe énorme. Parfois la nuit il voulut chasser à la lumière
les papillons nocturnes. Il en fut rudement réprimandé et il dut se
rabattre sur les chasses diurnes et crépusculaires. Tout d'abord nous
avions ri de ce goût-là. Nous finîmes par le respecter, nous persuadant
que Tabourin était un homme de science, peut-être un grand
entomologiste.

Un jour, nous eûmes à faire une reconnaissance périlleuse dans un
terrain occupé par l'ennemi. Nous étions douze et un lieutenant. Nous
parcourûmes tantôt en nous cachant comme des lapins, tantôt en faisant
des sauts de chèvre, une assez grande étendue sans nous laisser
découvrir. Nous sortions d'un bois quand on s'aperçut qu'il nous
manquait un homme. Cet homme c'était Tabourin. Étonné, car nous
n'avions entendu aucun coup de fusil, le lieutenant s'arrêta et commanda
à deux soldats de retourner sur leurs pas pour le chercher. Ils
revinrent bientôt sans l'avoir découvert. Nous continuâmes de
reconnaître le terrain en nous couvrant plus soigneusement aux regards:
nous étions en plein dans les lignes ennemies.

Tout à coup, au moment de descendre dans une dépression du terrain, nous
apercevons au-dessous de nous deux soldats qui parlaient avec animation:
un Allemand et un Français. A notre vue le premier prit la fuite. Le
lieutenant, croyant logiquement qu'il s'agissait d'un espion, commanda
le feu, sûr en même temps que nous nous découvrions du coup. L'Allemand
n'avait pas fait vingt pas qu'il tombait criblé de balles.

Fou de fureur, le visage injecté, notre lieutenant s'avança vers
Tabourin, le revolver au poing.

-- Ah, sale bête! Traître!

Tabourin laissa tomber son fusil et, l'air extraordinairement
tranquille, ouvrit les bras pour recevoir le coup. Le même sourire
mystérieux et sardonique lui contractait les lèvres.

Il reçut la décharge en pleine poitrine. Il tomba de tout son long, les
bras toujours ouverts, comme s'il eût voulu étreindre cette terre qu'il
aimait tant.

Nous étions découverts; nous fûmes poursuivis de près; on perdit trois
hommes; je fus blessé. Je parvins néanmoins à me traîner jusqu'à nos
tranchées, où je fus secouru.

Quelques jours après, ajouta l'aimable invalide en souriant, ma pauvre
jambe allait pourrir dans le cimetière du village où l'on avait établi
notre ambulance et moi je m'en revins ici avec mes histoires
militaires.

-- Mais êtes-vous sûr, vous, que Tabourin trahissait? demandai-je ému par
ce récit.

-- Je suis sûr de tout le contraire. Pour moi, le soldat allemand était
un entomologiste comme lui. Ils s'étaient rencontrés l'un l'autre en
poursuivant un insecte quelconque et ils étaient sans doute entièrement
occupés de leur science quand nous leur avons tombé dessus.

\end{chapter}

\begin{chapter}{Les deux idéals}

Depuis la chute de l'Empire d'Occident l'Europe n'a pas traversé de
moments plus critiques que ceux-ci. Le commun s'imagine que cette guerre
est une guerre de commerçants: il ignore que son véritable objet est le
concept de l'État et le concept même de la vie.

Ce qui est en lutte présentement, ce sont deux idéals: l'idéal germain
et l'idéal latin. Le premier, nourri en d'autres temps par le panthéisme
idéaliste, tombé ensuite dans le pessimisme et enfin dans le monisme
matérialiste, est aujourd'hui franchement antichrétien. Les directeurs,
il est vrai, invoquent le nom de Dieu; mais, qu'on y prenne garde, ce
Dieu est un Dieu allemand avec un État-major infaillible et une
artillerie lourde: un nouveau Jéhovah, qui se délecte des cris de
douleur poussés par les ennemis de son peuple.

La morale germanique, d'accord avec la pensée de Frédéric Nietzsche, son
dernier philosophe, a renversé l'ancienne échelle des valeurs. Les bons,
ce sont les forts; les mauvais, les faibles. Nous ne devons obéir qu'à
un instinct primordial: l'instinct d'accroître ses forces. Voilà la loi
fondamentale de l'existence. La morale est une invention des hommes;
Dieu, le Bien, la Vérité, des fantômes issus de notre imagination. Il
n'y a qu'une réalité naturelle: la vie. L'individu sain et fort, et qui
aime la vie, est seul digne de vivre. Celui qui s'enquiert du bien et de
la vérité pour eux-mêmes et non par amour de la vie, celui-là est un
dégénéré.

Et qu'on ne croie pas que ces principes se trouvent dans tel ou tel
penseur isolé de l'Allemagne. Voilés ou découverts, ils paraissent dans
la plupart des livres publiés là-bas depuis quelques années. Lisez
attentivement le manifeste par lequel les intellectuels allemands ont
prétendu excuser l'invasion de la Belgique et la destruction de ses
cités, et vous les y verrez palpiter.

Le concept germanique de l'État répond à ce concept de la vie. De même
que l'individu doit subordonner tous ses instincts au primordial
instinct d'accroître ses forces afin que la vie soit de plus en plus
exubérante, de même la totalité de ces individus doit se subordonner à
la vie de l'État afin qu'il soit de jour en jour plus fort, plus apte à
dominer. C'est la résurrection de l'idée spartiate. Les nations sont
comme les individus: les uns sont dignes de vivre, les autres peuvent
disparaître. Nous, dont l'instinct vital s'est amorti, nous les Latins,
nous sommes des décadents, des impuissants et nous devons livrer passage
à la race germanique, dont la vie, sans cesse en progrès, figure ce
qu'il y a de plus haut, de plus splendide dans l'humanité.

Que les germanophiles espagnols ne s'y trompent pas: ce dont ils se
plaignent, c'est de quelques blessures que la vanité française leur a
faites; ce sont des jalousies, des querelles de frères. Mais le mépris
allemand est bien plus sincère et par conséquent plus humiliant.
L'Allemagne contemple notre Espagne avec la froide attention du
naturaliste qui examine un insecte.

Je ne veux cependant pas commettre l'injustice de supposer que tous les
Allemands partageant ces idées-là. J'ai parmi eux de bons amis qui les
détestent autant que moi. Mais il faut aussi reconnaître qu'elles sont
très répandues chez eux et déclarer surtout que les grands hommes de
l'Allemagne, aussi bien les hommes d'action que les intellectuels, les
approuvent et les célèbrent ouvertement ou secrètement.

Nous avons l'habitude de ne regarder que la glorieuse Allemagne de la
fin du dix-huitième, l'empire alors des grandes idées et des sentiments
nobles. Quand on se rappelle cette époque, la mémoire s'emplit des noms
de Gœthe, de Schiller, de Lessing, de Wieland, de Kant, de Fichte, de
Schelling, de Richter, et nous nous représentons cette petite et
éminente société qui ressembla tant à celle d'Athènes. Mais, las! que
l'Allemagne d'aujourd'hui lui ressemble peu! Elle a des savants
considérables, de consciencieux chercheurs, mais des poètes et des
métaphysiciens inspirés, non. La science semble y être subordonnée à
l'industrie, la philosophie à la gloire militaire.

Je me souviens qu'au lendemain de sa résonnante victoire sur la France
(j'étais encore un enfant), je visitai avec mon père une grande fabrique
espagnole dans laquelle il y avait des ingénieurs allemands. On était à
table, le repas achevé, quand un des ingénieurs (il s'appelait Jacobi,
comme l'aimable philosophe ami de Gœthe) se mit à dénombrer avec une
orgueilleuse satisfaction les produits que son pays fabriquait et
exportait aux autres. Sa longue liste terminée, il fit une pause, puis
ajouta en souriant: «Et enfin la philosophie, que nous exportons aussi.»

Qu'est-ce que cela signifie, sinon que les Allemands ne considèrent plus
leurs philosophes que comme de vénérables ruines bonnes à exciter les
étrangers curieux!

De même que les Japonais ne croient point en leurs idoles, les Allemands
ne croient point en leurs philosophes. Ils les montrent en souriant aux
touristes, les portent aux autres nations, comme nous les Espagnols nos
chanteurs «flamencos».

Latins, Slaves, Anglo-Saxons, en retard sans doute dans l'évolution
biologique, nous n'avons encore pas atteint la sérénité olympienne qui
caractérise les Germains de nos jours. Leur empereur n'est pas ému par
la pensée des milliers d'hommes qu'il envoie quotidiennement à la mort.
Si devant ces champs de bataille où le sang ruisselle, nous nous sentons
saisis d'une infinie mélancolie, lui, l'Empereur, semblable à Jupiter,
père des Dieux, redresse sa moustache parfumée et sourit à notre
faiblesse puérile. Ses généraux, olympiens de second rang, ont observé
que la guerre est une nécessité biologique et le seul moyen d'empêcher
que la race des éphémères ne dégénère.

Vieux latins, nous continuons de penser que c'est pour eux-mêmes qu'il
faut rechercher la vérité et le bien, et non pas pour accroître notre
vitalité. Chez nous, les incrédules mêmes sont chrétiens, car aucun de
nous ne doute que la charité est la plus haute des vertus. Nous pensons
que le respect des faibles, la pitié, la compassion ne sont point des
sentiments qui débilitent, mais qui réconfortent, et que ce qui fait
vraiment dégénérer les hommes, c'est le pouvoir sans bornes. Tibère,
Néron et Domitien, trois monstrueuses hontes du genre humain, étaient de
très bonnes personnes avant de monter au trône.

Enfin, même si les Germains venaient à triompher, l'idéal chrétien ne
périrait point pour cela. Car les «portes de l'enfer ne prévaudront
jamais contre lui». Il subirait seulement une éclipse.

Pour soutenir leur hégémonie, non seulement l'Allemagne et l'Autriche
seraient dans la nécessité de poursuivre leurs armements et de rester
sur le pied de guerre, mais elles devraient en outre s'opposer par la
force à l'armement des autres nations. Nous serions trois cent millions
d'Européens réduits au même état où se trouvaient les Chinois en même
nombre quand, au treizième siècle, quelques tribus guerrières de la
Mongolie s'emparèrent de l'empire. Les empereurs mongols respectèrent
les coutumes des Chinois, mais ils leur interdirent les armes. Au bout
d'un siècle à peu près, les vaincus tramèrent un complot ténébreux,
quelque chose d'invraisemblable, et, le jour fixé, égorgèrent les
petites garnisons de soldats que les Mongols entretenaient dans les
villes de l'empire.

Nous autres, nous n'aurions même pas ce moyen-là: comment trouver en
Europe la dissimulation et le secret nécessaires à une conspiration de
cette taille?

Éloignons de nous ces visions d'Apocalypse qui ne se vérifieront
jamais. Pensons plutôt qu'après cette copieuse saignée et le jeûne
régénérateur auquel elle s'est soumise, l'Allemagne recouvrera la raison
et redeviendra, pour son propre bonheur, une nation tranquille avec des
philosophes, des poètes et des musiciens comme ceux que nous n'avons
cessé d'admirer.

\end{chapter}


\begin{chapter}{L'idole scientifique}

\begin{flushright}
Cap-Breton-sur-Mer, $28$ août $1916$.
\end{flushright}

La vieille histoire que nous avons apprise enfants, d'un peuple
cheminant dans le désert guidé par un nuage de feu, cette vieille
histoire est le symbole de la marche de l'Humanité sur la terre.

Vous rappelez-vous combien de fois, se détachant du seul vrai Dieu, ce
peuple tourna le dos à son chef et se laissa tomber dans les bras d'une
immonde idolâtrie? Suivez les pas du genre humain à travers l'histoire
et vous verrez le même acte attristant de déloyauté se reproduire sans
cesse. Le fanatisme, la superstition, l'idolâtrie nous épient toujours
dans notre pérégrination et nous tendent des lacs que nous ne pouvons
éviter.

La présente guerre a mis en évidence l'un des plus funestes de ces lacs
où soit tombée notre pauvre Humanité.

Certes nous les admirons, ces savants qui nous parlaient des molécules
comme s'ils eussent dansé toute la vie avec elles, qui nous en contaient
les secrets les plus intimes et, comme le serpent du Paradis, nous
laissaient entrevoir à travers de fallacieuses paroles que le jour était
proche où toute la science du Bien et du Mal nous appartiendrait.

Mais qui donc se souvient de Dieu! qui parle d'immortalité! Ouvrez un
des livres allemands de ces dernières années et, au milieu de
minutieuses analyses consacrées à quelque particularité de la science,
vous surprendrez une attaque intempestive, furieuse, contre ce que ces
savants appellent la «dégradation théologique», une flambée de haine
contre la superstition théiste.

Il n'y a qu'une seule divinité: la Vérité scientifique. Si au lieu
d'avoir un culte et de l'adoration pour elle, nous courons nous
prosterner devant les autels du vétuste Dieu de nos pères, les savants
modernes nous menacent d'éternelle condamnation intellectuelle. Le
magnifique édifice des sciences physiques doit remplacer le monument
ruineux de la théologie. Toutes nos croyances et tous nos espoirs sont
de pur subjectivisme. Il faut se garder de la foi comme d'une maladie
contagieuse. Croire en quelque chose qui ne soit pas évident à la
raison, c'est pécher ouvertement contre elle. Avoir foi en Dieu et dans
l'immortalité sans aucune preuve qui justifie cette foi, c'est se donner
un plaisir coupable, c'est d'une immoralité profonde.

Le vieil Hœckel, le savant le plus fameux de l'Allemagne moderne,
nous convie à adorer l'éther cosmique. Tout en sort, tout y rentre.
Agenouillons-nous et chantons: «Saint, immortel Saint!»

Pourquoi se moquer alors de ces pauvres nègres qui adoraient les
oignons? Il s'accomplit dans un oignon d'admirables et mystérieuses
opérations chimiques, que répètent celles de l'éther cosmique. Mieux
encore, l'éther impalpable, indivisible, s'y rencontre tout entier.

Nous autres hommes, il semble que l'ivresse nous attire d'une façon
irrésistible. Les limites nous indignent. Nous ne sommes contents que si
nous avons tout épuisé. Qu'est-ce que la scholastique, sinon l'ivresse
produite par la logique? N'est-ce pas une ivresse égalitaire que la
Révolution française? Le romantisme est-il autre chose qu'une ivresse
sentimentale? Vivons donc en pleine ivrognerie scientifique.

Il faut chercher la technique: la technique avant tout. Les
Mathématiques pures nous donnent la technique de la mesure; la Physique,
la technique des machines; la Chimie, celle des prodigieuses
transformations de l'industrie. La connaissance scientifique des
mœurs nous donnera une morale scientifique. La morale traditionnelle
est morte. A sa place reste la morale technique.

Tout le monde civilisé participe aujourd'hui à cette ivrognerie
technique. Mais ce sont les Allemands qui s'y sont principalement
adonnés. Et ils ont montré que leur vin était pire que celui de tous les
autres.

C'est un fait à peu près constant que l'alcool produit une
transformation du caractère. Un homme ordinairement taciturne,
insociable, devient, quand il a ingéré une raisonnable quantité de vin,
un joyeux compère, tout tendresse et affection, qui vous embrasse, vous
manie et vous laisse les épaules pleines de larmes et de bave.

Au contraire, les sujets les plus timides et les plus inoffensifs y ont
à peine goûté qu'ils acquièrent une humeur belliqueuse, impatiente,
montrent les poings et défient tout le monde.

Et voilà justement ce qui est arrivé aux nations. La France, qui a
toujours été un pays guerrier, s'est transformée sous l'influence de
l'ivresse scientifique en un pays humanitaire et pacifiste. L'Allemagne,
cette simple et bonne Allemagne des débuts du dix-neuvième siècle qui
faisait verser des larmes de tendresse à la sensible Mme de Staël, est
devenue agressive et provocante.

Cette radicale transformation me remet en mémoire le cas d'un de mes
condisciples d'Institut. Dans les premières années c'était un garçon
très appliqué, exact, pacifique, un étudiant modèle. Il évitait avec
soin les disputes. Si quelques-uns d'entre nous en venaient aux mains,
on le voyait devenir grave et s'éloigner le plus possible du théâtre des
coups.

Un jour, quelques minutes avant d'entrer en classe, un élève turbulent
et hargneux, le pire de nos compagnons, un garçon que nous craignions
tous, se mit à le railler de la plus féroce façon. Et non seulement il
l'abreuva des plus grossiers sarcasmes, mais en venant aux voies de
fait, il lui jetait à terre son chapeau chaque fois que l'autre le
remettait. Nous assistions à la scène, les uns non sans peine, les
autres avec gaieté, chacun selon son cœur. Le malheureux garçon,
silencieux et pâle, reprenait son chapeau par terre et tentait de se
retirer; mais l'autre, qui ne l'entendait pas ainsi, renouvelait sa
plaisanterie avec un plaisir grandissant. A la fin nous le vîmes si
blême que nous en fûmes effrayés. Il s'élança tout à coup sur son
agresseur avec une telle impétuosité qu'il le renversa sur le sol, puis,
lui montant dessus, lui appliqua de si rudes coups de poing sur le
visage qu'il le mit bientôt en sang.

Peu de jours après, sans aucun motif apparent, il défiait un des autres
querelleurs de la classe et le battait également. Dès lors, ce garçon si
docile, si aimable, devint, sans cesser de s'appliquer à l'étude, un
bravache insupportable et nous fûmes forcés de le fuir.

Eh bien, c'est quelque chose de semblable qui est arrivé aux savants à
lunettes de l'Allemagne. Il n'y a rien de plus détestable qu'un
pacifique converti du soir au matin en fier-à-bras.

Il s'est produit, il y a quelques jours, dans cette région tranquille,
une singulière alarme. Le bruit avait couru dans le village qu'un homme
suspect traversait le bois à bicyclette et l'on disait que c'était un
prisonnier évadé.

Le téléphone commença de fonctionner d'un centre à l'autre. On annonça
enfin son passage dans un village voisin et un groupe d'habitants partit
dans le dessein de l'arrêter. Ils y réussirent. Le fugitif était en
effet un officier allemand; il était en bras de chemise, portait des
lunettes (cela va de soi) et avait une fine tête intelligente.

Il se laissa prendre sans résistance. On le conduisit à la mairie, où,
poussés par la curiosité, nous nous rendîmes aussi. Il parlait
correctement le français et assez bien l'espagnol. Nous lui adressâmes
la parole tandis qu'arrivaient les gendarmes qu'on était allé chercher,
et il nous répondit avec cette froide hauteur et ce ton de supériorité
si fréquents aujourd'hui chez les Germains. Car ils sont arrivés à se
persuader qu'il n'y a de science, de culture, de bon sens même, que
dans la seule Allemagne.

Une des personnes qui se trouvaient là osa discuter avec lui les fins de
la guerre. Le prisonnier n'hésita pas à déclarer que la victoire de
l'Allemagne était certaine et que le genre humain y gagnerait beaucoup.

-- Sur quoi vous basez-vous pour supposer ce dernier point? lui
demandai-je, piqué de curiosité.

-- Sur ce que l'Allemagne, répliqua-t-il, est le seul pays actuellement
organisé. Il existe dans les autres pays des éléments de culture
assurément très considérables, mais épars. Il leur manque cette efficace
unité sans laquelle le plus souvent ces éléments demeurent stériles.
Dans la guerre comme dans la paix, dans les sciences comme dans les
arts, ce qu'il vous faut, c'est une cohésion, une discipline que la
prépondérance de l'Allemagne est seule capable de donner. Vous ne
pouvez pas voir les choses d'une façon continue et intellectuelle, ni en
donner la véritable explication scientifique, car vous travaillez sans
ordre. Ce sont des efforts isolés, subjectifs, des produits de
l'initiative individuelle, qui n'engendrent que des résultats
superficiels.

-- Ces efforts isolés, répartis-je, ont pourtant produit toute la science
et tout l'art qui aient été et qui soient sur notre planète. Ni Platon,
ni Aristote, ni Shakespeare, ni Cervantés, ni Képler, ni Galilée n'ont
eu besoin de votre organisation de fer pour arracher à ce monde des
trésors de beauté et de vérité. Que signifie cette discipline
scientifique? Voudriez-vous par hasard que des poètes et des savants se
missent en uniforme? Quel avantage y aurait-il que Pasteur se fût mis à
ses expériences sur un coup de trompette ou qu'Anatole France eût écrit
ses ouvrages par ordre du général-commandant la région?

Les yeux du prisonnier étincelèrent de colère comme si on l'eût pincé,
et il me fit entendre dans des termes peu polis que j'étais d'autant
moins autorisé à lui tenir tête que j'étais Espagnol.

Il continua de causer avec les autres personnes. Sans doute ne lui
agissaient-elles pas autant que moi sur les nerfs. Cependant, comme
l'une d'elles reprochait aux Allemands les actes de cruauté qu'ils
avaient commis en Belgique et dans le nord de la France, il répliqua
avec un sourire sarcastique:

-- Ce reproche indique qu'il n'y a pas encore en France un esprit
vraiment scientifique. Pour déterminer le bien et le mal des choses, il
est nécessaire de fuir les concepts à priori et de bien comprendre que
tout, absolument tout, dépend des résultats expérimentaux. La discipline
scientifique nous oblige de penser que, seule, une systématisation des
faits nous donnera la vérité exacte, et jamais les spéculations de
l'imagination individuelle. Pour vous la guerre est une aventure; c'est
pour nous un théorème. Nous considérons le résultat et nous le
développons d'une manière inflexible. La guerre la plus cruelle est
nécessairement la plus courte.

-- Je me félicite vivement, m'écriai-je, de n'être pas un homme de
science! Mieux vaut mourir dans une ignorance crasse que de porter une
conscience chargée de cruauté. Nous tous ici, nous sommes des chrétiens
et nous voyons en chacun de nos semblables l'image de Dieu et non point
des moutons ou des bœufs qui doivent être sacrifiés pour que les
autres existent. Le plus grand philosophe que vous ayez eu, Emmanuel
Kant, a dit admirablement que nous «devions toujours prendre un être
humain comme fin et non point comme moyen».

-- Ce sont des subtilités de philosophes, des vieilleries métaphysiques,
qu'aucun esprit positif ne peut croire de nos jours, répondit-il sans
cesser de sourire. Nos actes de cruauté ont été et sont absolument
nécessaires comme les termes d'un théorème, et ils ont une explication
satisfaisante parce qu'elle est scientifique.

-- Vous voulez dire que ce sont des assassinats scientifiques?

Il me jeta un long regard de colère et de mépris, et me tourna le dos.

Je n'en fus pas le moins du monde touché. La seule chose qui
m'affligerait, c'est que des hommes honorables et pitoyables me
tournassent eux aussi le dos.

De cet entretien, comme d'ailleurs de tout ce que je lis et observe,
j'ai tiré la conviction que les Alliés n'obtiendront rien de tels
hommes en leur enlevant des canons, s'ils ne leur enlèvent auparavant
leurs idées.

\end{chapter}


\begin{chapter}{La religion de la France}

L'irréligion de la France est le topique dont ses ennemis ont tiré le
plus de profit. Un moine à qui je faisais part en Espagne du grand
mouvement religieux qui s'est produit en France, me disait:

-- C'est possible: les Français se souviennent de sainte Barbe quand il
tonne.

-- Est-ce que par hasard les Espagnols se souviendraient d'elle quand le
ciel est bleu? répliquai-je. Je crois bien que nous ne pensons presque
tous à l'autre monde qu'au moment de prendre congé de celui-ci, à moins
que des parentes ou des voisines ne nous aient glissé un prêtre dans
notre chambre et dit avec plus ou moins de ménagements: «Tu vas mourir:
prépare-toi!»

-- Oh, mais chez nous les églises sont pleines de monde, grâces à Dieu!

-- Pleines de femmes, oui. Le matin, à l'église, je n'ai jamais vu qu'un
seul homme aller à la sainte table pour trente ou quarante femmes qui y
allaient. En Espagne, on dirait que nous chargeons les femmes de notre
religion, comme elles sont chargées de notre cuisine et de notre
blanchissage.

Il faut reconnaître qu'elles s'acquittent de la première de ces charges
avec une diligence et une perfection qu'elles ont peu l'habitude
d'apporter dans la seconde. C'est vraiment étonnant que de voir l'ardeur
avec laquelle quantité de femmes accourent au temple à toutes heures du
jour! J'en suis arrivé à m'imaginer que pour certaines âmes timorées
Dieu est une sorte de Louis XIV qui a constamment besoin d'être adulé.
Elles courent à la neuvaine et aux quarante heures, comme les courtisans
se pressaient à Versailles au «dîner» et au «coucher» du roi. Je connais
une dame qui va toujours communier avec trois ou quatre scapulaires
pendus au cou. S'il lui arrive d'en oublier un chez elle, ce n'est qu'en
tremblant qu'elle s'avance vers la sainte table, craignant que Notre
Seigneur ne lui en veuille de ne se point présenter avec toutes ses
décorations.

Mais les esprits qui prennent la religion au sérieux observent avec
chagrin qu'il y a peu de gens qui ont une foi vraie et claire. Nous
avons l'habitude d'attribuer ce fait à la corruption des temps; c'est
une erreur. Bien des personnes s'extasient sincèrement en parlant de la
ferveur des temps anciens, et pourtant, alors comme aujourd'hui, les
âmes soucieuses des choses éternelles étaient en très petit nombre. La
dévotion était plus apparente, il y avait plus d'hypocrisie. On aimait
plus la terre que le ciel.

\horizontalLine

En réalité, que ce soit avant ou après Jésus-Christ, les hommes se sont
toujours divisés en païens et en chrétiens. Les premiers supposent que
nous avons été mis au monde pour jouir; les seconds croient que nous
sommes nés pour travailler et souffrir. Il s'agit là uniquement de la
façon dont on conçoit la vie. César Borgia, bien que cardinal de
l'Église catholique, était païen et un vrai païen, et son méchant
entourage l'était aussi, et aussi toute la Cour du pape Alexandre VI et
les cardinaux qui mangèrent cent plateaux de confiseries aux noces de
Lucrèce Borgia et dansèrent avec leurs dames et avec celles de la
princesse de Squillace, comme le rapporte une lettre récemment
découverte par notre savant compatriote le marquis de Laurencin. Mais
Socrate, Léonidas, Régulus, Sénèque, les Gracques, Pauline, Térence et
tous les martyrs ignorés de l'antiquité, dont les noms ne sont pas
arrivés jusqu'à nous, étaient des chrétiens. Il ne faut pas oublier la
belle sentence de saint Anselme: «Le Christ étant la vérité et la
justice, quiconque meurt pour la justice et la vérité, même s'il ne
croit pas au Christ, meurt pour le Christ.»

Mais il y a de suprêmes instants dans la vie où ces païens peuvent
devenir des chrétiens. Nous naissons tous imprégnés de foi. Dès qu'une
petite porte s'ouvre dans notre cœur, la religion s'y précipite.
C'est pourquoi nous voyons nombre de grands pécheurs se convertir sous
le coup de la foi en chrétiens fervents. Cette même Lucrèce Borgia dont
nous parlions tout à l'heure menait une vie exemplaire à Ferrare dans
les dernières années de sa vie. Elle portait sans cesse un cilice; elle
laissa à sa mort la réputation d'une sainte.

Il faut toutefois pour cela que le cerveau n'ait subi aucune diminution.
Si singulier que cela paraisse, les blessures du cœur se guérissent
plus facilement que celles de la tête. Quand la cervelle se gâte, il n'y
a plus de remède pour le malade. Car, aujourd'hui comme toujours, ce
sont les idées qui gouvernent le monde. Les idées engendrent les
sentiments et les actes, ou, ce qui est la même chose, toute la vie de
l'homme. Nous ne sommes pas ce que nous sentons, mais ce que nous
pensons; nous sommes toujours proportionnés à nos idées, et notre âme
s'abaisse ou s'élève à mesure que s'élève ou s'abaisse notre état
mental.

Aussi se trompe-t-on fort quand on pense que les idées n'ont aucune
influence sur la conduite de l'homme; mais on se tromperait bien plus
encore si l'on croyait, comme au moyen âge, qu'elles ne doivent
s'inculquer que par le feu et le marteau.

\horizontalLine

Telle est, sur ce terrain, la situation qu'occupe la France vis-à-vis de
l'Allemagne. Les Français sont des pécheurs: j'ai donné précédemment mes
raisons de penser ainsi. Ils avaient dans une certaine mesure le cœur
égaré. Les Allemands sont des philosophes: ils ont le cerveau corrompu.

Ce n'est pas parce qu'elle a expulsé les ordres religieux qu'on peut
dire de la France qu'elle a perdu sa religion. L'Espagne a-t-elle perdu
la sienne quand notre roi catholique, Carlos III, chassa plus
cruellement encore la Compagnie de Jésus, et plus tard, quand notre
Gouvernement décréta la suppression de tous les moines et que,
pénétrant dans les couvents, la populace en égorgea les occupants? Ces
décisions n'ont rien à voir avec la religion des pays où elles sont
appliquées. Parcourez les départements français, visitez-en les villages
et vous y trouverez exactement reproduit le type de notre religiosité
espagnole. C'est que le catholicisme, ainsi que son nom l'indique, a eu
la vertu d'unifier tous les hommes, de leur mettre son timbre, en les
rendant semblables entre eux devant l'autel. Ce sont les mêmes
solennités, les mêmes processions, les mêmes Confréries, les mêmes fêtes
profanes ne faisant qu'un avec les fêtes religieuses. Les petits enfants
suivent le catéchisme, les jeunes filles assistent aux processions avec
leur médaille et sous le voile blanc des filles de Marie; les vieilles
femmes vont infailliblement aux offices de l'après-midi. La première
communion se célèbre en France avec une pompe et une allégresse comme
je n'en ai jamais vu en Espagne. Les parents viennent de loin à cette
occasion, comme on fait chez nous pour un mariage; la maison se
transforme en un temple, la rue est jonchée de fleurs. Au grand ennui
des confesseurs, mais à la grande joie des sacristains, le type
classique de la bigote est lui aussi représenté à la fête.

D'où vient donc cette haine à mort pour la nation française? Quelle est
la folie qui a frappé tant de catholiques et un assez grand nombre de
prêtres? J'ai entendu l'un de ces derniers prononcer la phrase suivante:
«Si la France se tirait victorieusement de cette guerre, je douterais de
l'existence de Dieu.»

Est-ce d'un chrétien? Est-ce même d'un homme?

On lit peu de livres allemands en Espagne; l'allemand est une langue
sans grande diffusion chez nous et ses traducteurs sont rares. Il faut
avouer d'ailleurs qu'en général ces livres sont une nourriture trop
forte pour nos estomacs de Latins. Aussi ne sait-on pas bien en Espagne
quel est l'état mental de l'Allemagne d'aujourd'hui. Mais il suffit
d'avoir suivi avec quelque attention l'histoire de sa philosophie
pendant les temps modernes pour voir que la religion de l'Allemagne
intellectuelle au cours de ce dernier siècle n'est point le
christianisme, mais le panthéisme. Le panthéisme ne saurait fonder la
morale: il la nie absolument. Il n'est par conséquent qu'un pont qui
conduit au monisme, et il y a beau temps que les intellectuels allemands
ont franchi ce pont-là. La théorie du surhomme et de la surnation,
théories dominantes aujourd'hui en Allemagne, découlent naturellement de
ce matérialisme.

Mais, dira-t-on, les intellectuels ne sont pas le pays. Grave erreur.
Les intellectuels sont toujours la nation présente ou future. Les idées
sont comme les cours d'eau; elles naissent sur les cimes. Mais elles
descendent peu à peu, en suivant le flanc des montagnes, jusqu'aux
bas-fonds ou bien s'infiltrent secrètement dans les terrains perméables
et vous trempent au moment qu'on s'y attend le moins. Presque personne
ne lit Platon et pourtant le plus rustre des hommes de nos jours est
imprégné de platonisme. Ainsi en est-il du peuple allemand: il ne lit
point Kant, mais il est pénétré jusqu'aux os de son «modeste athéisme»,
comme disait Coleridge. Les Allemands sont hégeliens sans avoir lu
Hegel, car les poètes, les dramaturges, les romanciers, les critiques et
les journalistes se sont chargés de leur servir avec d'appétissants
assaisonnements le plat du fatalisme panthéiste.

Mais l'Allemagne n'aurait-elle point la foi? Oui, elle a la foi, elle en
a même beaucoup. Mais c'est dans la chimie. Dieu y est transformé en
machinerie, en charbon, en électricité. Il n'est pas venu au monde pour
souffrir et mourir: il y est venu pour vivre et faire souffrir. Soyons
puissants, triturons nos voisins, imposons partout notre volonté, et la
Divinité paraîtra en nous ce qu'elle est: une force immanente et
universelle.

Quelques catholiques espagnols s'attendrissent en lisant à chaque pas le
nom de Dieu dans les proclamations du Kaiser et de ses généraux. Ils
sont victimes d'une admirable falsification: ce Dieu a lui aussi été
extrait du charbon, comme maints autres produits extraordinaires.

Mais le vrai Dieu, le Dieu légitime a une expérience infinie de ces
affaires de psychologie et ne se laisse pas tromper par les marques de
fabrique allemande. Il voit «made in Germany» sur l'étiquette et
repousse l'article, tout en reconnaissant qu'il est bien présenté.

\horizontalLine

L'esprit gaulois n'est pas panthéiste. Du moins ne l'est-il plus depuis
le jour lointain où le christianisme tua le druidisme dans les bois de
la Gaule. L'idée que les Français se font de la divinité, soit pour
l'affirmer, soit pour la nier, est la vraie. Il y a parmi eux un assez
grand nombre de sceptiques, Montaignes en miniature; il y a bien plus de
Rabelais passionnés de bonne chère et de vin; mais on ne trouverait pas
dans toute la France un Frédéric Nietzsche, ou quelqu'un qui fût capable
de soutenir le mal par principe.

Dans chaque pays, comme dans chaque homme, la foi et le scepticisme sont
des états instables qui se succèdent. Il ne faut pas trop donner
d'importance à ces fluctuations. Elle tiennent à l'imperfection même de
notre nature et il faut s'y résigner. Les arbres sont tour à tour vêtus
de feuilles et tout nus. Qui eût dit qu'après le sceptique dix-huitième
siècle dût se lever le spiritualiste dix-neuvième, qu'après Voltaire,
Diderot et Helvétius paraîtraient Chateaubriand, Lamartine, de Bonald et
de Maistre? Ce qui est très important, c'est la substitution d'une foi à
une autre, et c'est ce qui arrive présentement en Allemagne.

Les Français ont commis récemment la même folie que nous avons faite il
y a quatre-vingts ans: la suppression des ordres religieux.

Ne parlons pas de la séparation de l'Église et de l'État. Bien des
catholiques refusent d'admettre que l'Église soit un organisme de l'État
et préfèrent l'indépendance absolue à un protectorat importun et
intéressé. Parlons seulement des ordres religieux.

Il est hors de doute que leur expulsion a été un fait arbitraire et
scandaleux. En interdisant les Congrégations, la République française
commettait une injustice horrible, portait atteinte à la liberté et du
coup dénonçait ses propres principes: liberté, égalité, fraternité.

Mais qu'il me soit permis de poser quelques questions aux Congrégations
expulsées. Ont-elles toujours examiné leur conscience à fond?
L'ont-elles examinée scrupuleusement? N'y ont-elles pas trouvé quelque
haine des institutions républicaines? N'ont-elles pas conspiré parfois
contre ces institutions?

Si elles ne se tirent pas de cet examen complètement exemptes de péché,
elles ne doivent pas s'étonner de la pénitence. Qui sème la haine ne
peut recueillir l'amour. L'abeille se nourrit de miel, la nature lui en
donne; la puce vit de sang, la nature lui fournit le moyen d'en avoir.
La nature nous pourvoit généreusement de ce que nous lui demandons.
C'est une loi, et une loi consolante.

Si les religieux français avaient accepté loyalement les institutions
républicaines, la République ne leur eût pas mis la main dessus. «Si tu
veux que les femmes te suivent, disait notre Quevedo, marche devant
elles.» Pourquoi ne pas accepter franchement la République? Le pape Léon
XIII, d'inoubliable mémoire, ne l'avait-il pas fait? Marcher devant les
hommes, voilà le secret de les guider.

\horizontalLine

Le Français n'est pas un impie-né, comme on le propage en Espagne, par
ignorance ou dans d'ignobles desseins. Comme tous ceux qui sont nés et
ont été élevés dans la foi du Christ, les Français gardent leur
religion dans l'âme comme un fonds de réserve. Quand ils sont heureux,
ils délaissent les pratiques religieuses; ils y recourent dans le
malheur et y puisent leur consolation. En Espagne, nous faisons
exactement la même chose. Sans douleur, point de religion.

J'ai vu s'emplir de monde certaines nuits une petite église de village.
De pauvres femmes en deuil y accouraient, tirant par la main des enfants
également en deuil. Des vieillards, le visage pâle et le regard triste,
les suivaient d'un pas chancelant. Et dans le silence auguste du temple,
tandis que les cœurs demandaient au Très-Haut sa miséricorde, de
temps à autre éclatait un sanglot dont j'avais les entrailles remuées.

A Paris, cette foule élégante qui en d'autres temps courait les lieux de
plaisir envahit aujourd'hui les églises. J'ai eu peine à trouver place
à Saint-Sulpice, à Saint-Germain-l'Auxerrois, à la Trinité, à
Notre-Dame-des-Victoires. Et vous n'y voyez pas que des femmes comme à
Madrid: il y a là des hommes, et qui prient avec autant de ferveur
qu'elles. Celui qui ne se sent pas pénétré de respect devant cette
humble foule affligée, qui à genoux aux pieds de la Vierge demande le
soulagement de ses peines, celui-là pourra se dire chrétien; mais qu'il
est loin d'en mériter le nom!

Et là-bas, au front, sur la ligne de feu?

Ah, là-bas, ce sont les scènes mêmes des Croisades qui se reproduisent!
Une compagnie de soldats attendant l'ordre de sortir s'est massée au
fond d'une tranchée. Les grenades tombent, éclatent avec un bruit
épouvantable; la terre se lève et se meut comme une mer en courroux; et
voici qu'arrivent les lignes serrées de l'infanterie allemande, poussant
devant elles les mitrailleuses, moissonneuses d'hommes. L'heure de
s'élancer à travers cet enfer a sonné. Les cœurs battent, les mains
tremblent, les gorges se nouent. Alors, à cette minute suprême, la voix
d'un pauvre soldat s'élève avec autorité: «Ceux qui croient en Jésus
crucifié, à genoux! Que chacun se repente de ses fautes; je vais donner
l'absolution.» Et tous tombent à genoux, et, levant son bras, le
prêtre-soldat les absout.

-- Jamais je n'oublierai cet instant, me disait le blessé qui me
rapportait ce trait.

-- Et vous aurez raison, lui répondis-je. Un pareil instant suffit à
ennoblir toute une vie.

Une autre fois, dans une reconnaissance, un soldat de la patrouille
tombe blessé. Un de ses camarades se précipite à son secours et essaie
de s'en charger pour le transporter à l'ambulance.

-- Ne t'occupe pas de moi, dit le blessé. Je suis perdu. Je vais
seulement te demander une chose. Je suis prêtre et je te prie
instamment, à la première occasion, de recevoir pour moi la communion.
Je n'aurai pas le temps d'avoir la consolation de recevoir mon Dieu.

Le camarade, confus, honteux, garde le silence un instant. C'était un
garçon riche, dissipé et qui depuis des années s'était tenu loin de la
religion. Il dit enfin:

-- Je ne me suis pas confessé depuis mon enfance, mais je ferai ce que tu
me demandes. Dieu m'a touché par ce que tu viens de me dire. Dans une
minute une balle me tuera peut-être moi aussi. Tu es prêtre,
confesse-moi.

Il fit ainsi l'aveu de ses fautes et son camarade moribond lui donna
l'absolution.

Quel tableau! On le dirait tiré de La légende dorée et tracé sur un de
ces manuscrits du moyen âge qu'illustrait la main pieuse des moines.

Ah, défaisons-nous des préventions injustes! Ne nous flattons plus tant,
nous Espagnols, d'être seuls religieux; ne critiquons pas trop le
voisin. Demandons plutôt au Ciel que quand viendra pour nous le jour de
la grande épreuve, nous sachions nous aussi montrer la même foi et le
même courage que la France.

\end{chapter}


\begin{chapter}{Et après?}

Et de cette guerre incroyable, de cette guerre comme on n'en a jamais vu
et comme on n'en reverra jamais plus, que restera-t-il? Tous ces
ruisseaux de sang féconderont-ils la terre qui les aura bus?
Sècheront-ils au contraire la racine des fleurs, et notre planète ne
sera-t-elle plus jamais qu'un sinistre enclos de douleur et d'épouvante?

Je ne suis ni optimiste ni pessimiste. Penser que la guerre est dans
l'ordre des choses créées et qu'elle est périodiquement nécessaire pour
tempérer les excès de la fécondité, c'est à mon sens un blasphème. Je
n'ai jamais cru à l'utilité du mal; je n'ai jamais cru que le mal venait
de Dieu. Notre liberté, qui est tout ensemble notre perfection et notre
imperfection, engendre toutes les dépravations que nous observons dans
le monde. Et Dieu même est impuissant contre notre liberté.

Mais s'imaginer que l'Esprit de Vérité et de Justice qui gouverne le
monde va se croiser les bras et ne tirera point parti pour notre bien de
nos erreurs et de notre méchanceté, c'est également blâmable.

Dans notre voyage sur terre, nous entassons sous nos pas
d'infranchissables obstacles; mais une main divine les éloigne de nous.
Nous semons des écueils à l'envi, mais il y a quelqu'un qui prend soin
de les retirer.

La présente guerre est un mal dont il naîtra quelque bien. Ne parlons
pas de races perdues, anéanties, qui n'ont fait qu'apprêter le terrain
pour de nouvelles races. Ne parlons pas non plus de vieux systèmes qui
se défont pour faire place à d'autres plus parfaits.

Ne disons pas que la férocité est nécessaire à l'équilibre de
l'existence et que la domination des plus forts est légitime. C'est un
langage d'impie, que je ne sais pas balbutier. Pensons plutôt que
l'homme n'a pas été fait pour la guerre, mais pour la paix; car il n'est
pas la continuation de l'animal, mais un saut hors de lui. Nous sommes
composés d'atomes brutaux; nous ne sommes pas un atome brutal. S'il
arrive qu'en nous le lion rugisse et que le vautour croasse, n'en soyons
pas inquiets: ils y sont comme en cage.

Les nations sont comme les individus: elles ont des accès périodiques de
colère. Les physiologues ont défini la colère une courte folie. Cette
folie nous laisse toujours quelque chose de mauvais dans l'organisme,
trouble l'équilibre de nos humeurs, cause des dommages à la machine
corporelle.

Mais ce qui se passe dans l'âme est différent. Quand nous nous
rétablissons d'une de ces fièvres mortelles, nous ne manquons jamais
d'éprouver quelque confusion, quelque honte. Cette honte, c'est la
reconnaissance de notre être spirituel, c'est la voix d'En-haut qui nous
montre notre destin. Nous courons à la cage des lions et des tigres, et
nous lui donnons un second tour de clef.

C'est la même chose qui arrive aux nations européennes. Après la colère
dont elles ont été prises, après cette formidable attaque de nerfs, des
jours de détente et de réflexion viendront, et ces nations se sentiront
profondément honteuses. Mécontentes d'elles-mêmes, elles fermeront les
yeux et méditeront longuement. Une grande réforme morale se prépare. Le
Droit international va faire un saut prodigieux.

Mais les villages dévastés?-- Ils se repeupleront: le grincement des
charrettes et le chant du paysan sonneront de nouveau dans les lieux que
remplissent aujourd'hui les cris de bataille et la voix du canon.-- Et
ces milliers d'êtres mutilés?-- Ils penseront, résignés, qu'ils ont livré
leurs pieds et leurs mains au fauve pour racheter ceux de leurs frères
et qu'ils ont maintenant enchaîné ce fauve pour toujours.-- Et ces
larmes, tout ce sang répandu?-- Les larmes, c'est la rosée des âmes: il
faut que nous pleurions pour croître. Quant au sang, il aura été le prix
de notre rédemption.

La France a fait une cruelle expérience; mais c'est cette expérience qui
la sauvera. Elle vivait dans la langueur d'un bien-être matériel sans
exemple dans l'histoire. Son idéal, c'était de jouir. Une sensualité
sage et réfléchie régnait dans toutes les villes et se répandait dans
les campagnes. Quand cela se produit, quand nous adulons notre corps,
l'âme, offensée, nous abandonne et nous nous convertissons en une statue
vivante, comme celle dont parlait Condillac. Il n'y avait en cela rien
de mauvais, mais seulement de la froideur. Les liens d'homme à homme
s'étaient amollis; chacun se regardait le ventre: je te respecte pour
que tu me respectes, et rien de plus.

Or, ces règlements de Police ne suffisent pas à l'âme. Les salles du
Commissariat et de la Préfecture sont trop froides pour elle. Nous ne
sommes pas nés, nous les hommes, que pour échanger des coups de
chapeaux. Il a fallu cette grande catastrophe pour que les Français
fissent quelques pas en arrière et corrigeassent la direction de leur
marche. Quand le malheur entre dans une maison, les frères qui vivaient
loin les uns des autres, se voyaient à peine, s'embrassent en pleurant.
La fraternité, qui s'était fort relâchée en France dans ces dernières
années, fleurit de nouveau et exhale d'exquis parfums. Il faut signaler
cet événement: c'est ce que la terrible inondation laissera de plus
heureux derrière elle.

Une autre chose encore lui sera profitable: le culte de l'austérité. On
commence à en voir maints témoignages. Les français n'ont jamais été des
viveurs dissipés: ce sont des viveurs ordonnés. Je veux dire qu'ils se
sont toujours accordé le plus de plaisirs possible, mais que ce n'était
jamais sans calcul. Aujourd'hui ils renoncent résolument aux plaisirs.
Vous les verrez le lendemain de la paix déployer une activité fiévreuse
pour cicatriser les blessures de la guerre, pour recouvrer leur ancienne
prospérité: ainsi les fourmis d'une fourmilière bouleversée.

La politique s'assainira aussi. Oui, la politique avait besoin de se
refaire. On se rappelle qu'il y a deux ans, se prévalant de la haute
position politique de son mari, une femme assassinait un journaliste
connu. Quand on apprit que cette femme venait d'être acquittée par un
jury libre, tous les hommes qui en Europe ont quelque sens moral
s'écrièrent: «Il y a quelque chose de pourri!» Tous nous vîmes voltiger
les corbeaux sur la chair en putréfaction. Il était temps d'arrêter la
gangrène par le bistouri et le cautère, et ce sont les Allemands que la
Providence chargea de l'opération. Ils se chargèrent aussi de battre la
cataracte de ces partisans aveugles qui ignorent la tolérance et la
justice. «Que les Barbares sont longs à venir! Que fait donc Attila?»
s'écriait un jour Ernest Hello, en contemplant la corruption du second
Empire. Et Attila vint en effet peu de temps après. Le voici maintenant
revenu. Ce n'est plus cette fois pour châtier la luxure, mais le
mensonge. Si la République Française ne fait pas honneur à sa devise
«Liberté, Égalité, Fraternité», à quoi sert-elle?

Mais la Providence divine a beaucoup plus à faire en Allemagne. Le grand
péché des Germains, c'est l'orgueil. Et l'orgueil est le plus grand
péché de l'humanité; c'est celui qui fait vraiment de nous des bêtes.

Dans sa superbe, le roi Nabuchodonosor mangea du foin comme un bœuf.
Ne tombons-nous pas tous à quatre pattes dès que la fumée nous monte à
la tête?

D'où vient aux Allemands leur orgueil? Il leur vient surtout des excès
de leur industrialisme. En voyant qu'ils peuvent jouer avec les atomes,
les escamoter, transformer les gaz en solides et soumettre les forces
naturelles à toutes sortes de services, les hommes s'enflent
extraordinairement. Les Allemands, dans cet ordre de choses, avaient
fait plus de progrès qu'aucun peuple; ils en furent pleins d'eux-mêmes,
et ils se mirent à considérer avec mépris ceux qui ne savaient pas faire
du pain de bois et à se croire les élus de Dieu.

Mais Dieu n'a pas besoin de boulangers. Quand les mages de Pharaon
eurent converti les verges en serpents, celle d'Aaron avala toutes les
autres. Pour beaucoup de gens la fin et le résumé de toute la
civilisation, ce sont les cornues, les alambics et les gaz inflammables.
Il en est qui tremblent d'émoi, font les yeux blancs, quand on leur
parle des tours de danse que les Allemands font exécuter à la matière.
Je leur répondrais que même si je les voyais transformer un palais en un
immense feuilleté, je n'en continuerais pas moins à admirer davantage un
dialogue de Platon ou un drame de Shakespeare.

Au temps où se réunissaient à Weimar des hommes comme Gœthe,
Schiller, Herder, Wieland, Kotzebue, des musiciens inspirés, des grands
peintres, des architectes, des savants, des acteurs, les Allemands
étaient bien plus admirables qu'aujourd'hui avec tous leurs canons et
leurs zeppelins. Mais ce n'est pas une chose à dire au vulgaire: il ne
se prosterne que devant les œuvres tangibles, comme si le monde moral
n'avait point le pas sur le monde matériel et l'invisible sur le
visible.

Le progrès qui ne consiste qu'à utiliser les forces de la nature pour
notre avantage est un progrès chimérique. Si l'homme ne progresse pas
moralement, au lieu de se tourner à son avantage ces forces finissent
par concourir à sa perte. Et c'est précisément ce qui vient d'arriver.
Quand verra-t-on la fin de cette grossière superstition de
l'industrialisme? Platon, Épictète, Sophocle, Cicéron étaient des hommes
fort civilisés; ils s'éclairaient pourtant à l'huile, et l'apôtre saint
Paul, qui n'était pas un sauvage, ignorait le bicarbonate de soude. Le
cœur de l'homme sera toujours plus intéressant que la nature.
L'acteur importe plus que les coulisses ou le décor qui l'entourent.

Sa superbe en déroute, l'Allemagne redeviendra grande. Quand le vent de
la fortune nous souffle dessus, quand nos affaires prospèrent, que nous
vivons au milieu des commodités et que nous sommes enfoncés dans la
richesse, c'est alors que nous courons le plus grand risque de perdre le
bonheur. La sage Providence qui nous garde nous ouvre brusquement les
yeux pour nous permettre de redresser nos pas.

Il est inutile que nos viles passions se cachent sous le manteau du
patriotisme. Le patriotisme se compose d'un centième d'amour, le reste
est fait d'orgueil. De même que la loi divine et humaine nous donne le
droit de défendre notre vie en tant qu'individus, de même nous avons le
droit de défendre notre indépendance nationale par la force. Hors de
cela, le patriotisme n'est qu'un orgueil collectif.

Ce n'est pas parce qu'ils appartiennent à une grande nation qu'un
Allemand ou qu'un Russe sont plus grands, plus savants, ou plus heureux
qu'un Suisse ou qu'un Hollandais. La grandeur d'un homme ne se mesure
pas au terrain qu'occupent ses pieds, mais à l'horizon que son regard
découvre. Un mendiant anglais est comme un mendiant espagnol, et de même
un savant.

Les Allemands avaient atteint un degré inouï de prospérité industrielle
et commerciale. Je ne sais si les hommes étaient plus heureux pour cela
en Allemagne que dans les autres pays. Quoiqu'il en soit, au milieu de
leur prospérité, le serpent tentateur leur souffla à l'oreille qu'ils
devaient manger le fruit défendu. Ce fruit, c'était la richesse et
l'humiliation de leurs voisins. Ils pensèrent que les lois naturelles
étaient inévitables, mais qu'on pouvait se soustraire aux morales:
profonde erreur. Chassés de leur paradis (si c'en est un) ils seront
demain affligés, défaits, ensanglantés. Il est vrai qu'ils ont fait bien
du mal aux autres. Mais y a-t-il un homme au monde qui s'en puisse
féliciter? Espérons qu'après une expérience si douloureuse ils iront de
nouveau chercher leur ciel non plus à l'usine Krupp, mais où ils l'ont
toujours eu: dans la modération, dans la sobriété, dans la vie
tranquille de la famille, dans les bibliothèques et dans les salles de
concert.

\horizontalLine

Et quelles seront, pour l'Angleterre, les conséquences de cette guerre?

Nulles. Les dards les plus acérés s'émoussent sur la peau de l'éléphant.
La Grande-Bretagne ouvrira son Grand-Livre, passera au «Doit» les hommes
et les bateaux perdus, à l'«Avoir» les colonies allemandes conquises;
puis elle le refermera et, le parapluie sous le bras, ira faire sa
promenade.

C'est une singulière nation que l'Angleterre. J'ai lu dans mon enfance
un roman de Jules Verne où l'on voit un Français obséquieux qui cherche
à flatter le capitaine du bateau où il est. Ce capitaine est Anglais et
le Français lui dit: «J'admire tellement l'Angleterre que, si je n'étais
pas Français, c'est Anglais que je voudrais être.» Le capitaine tire une
bouffée de sa pipe et lui répond tranquillement: «Eh bien, moi, si je
n'étais pas Anglais, je voudrais le devenir.» Combien d'hommes en Europe
pensent de même!

J'admire la littérature, la politique, les mœurs, les jeux,
l'originalité de l'Angleterre. Je lui passe même son orgueil, qui n'a
rien d'agressif. Mais ce qui fait surtout que je l'admire, c'est qu'elle
est la patrie des hommes libres. Comparés aux siens, les hommes des
autres pays ne sont que des esclaves. Que de fois, en constatant
l'arbitraire et la violence du pouvoir en Espagne, en entendant parler
de l'insolence des militaires allemands, de l'intolérance des jacobins
français, de la cruauté des sbires russes, que de fois me suis-je dit:
«Prohibez, violentez, maltraitez: tant que l'Angleterre sera là, la
liberté du monde ne sera pas près de disparaître! C'est là qu'à la
dernière extrémité, nous qui ne sommes pas nés serviles, nous irons
chercher un refuge!»

On critique l'orgueil britannique. Et pourtant, partout où il y a
quelque chose d'admirable on rencontre un Anglais. Leur orgueil signifie
confiance en soi-même, et cela n'est pas pour inspirer de l'aversion
mais du respect. Quand éclata cette guerre, l'Europe croyait unanimement
que les immenses et lointaines colonies anglaises se lèveraient et
secoueraient le joug de leur domination. Les Allemands y comptaient
beaucoup aussi. C'est le contraire qui arriva. Les colonies se sentirent
blessées dans la métropole comme dans leur propre cœur et
s'apprêtèrent à porter secours à la mère-patrie.

On n'a pas assez médité ce fait, qui est unique dans l'histoire. Combien
il faut avoir été bon et généreux pour que ceux qui se trouvent dans
notre dépendance ne profitent pas des circonstances pour rompre soudain
avec nous! Il est possible que des actes de cruauté aient été commis
autrefois. Ils n'étaient d'ailleurs ni aussi nombreux ni aussi grands
que ceux qu'on reproche aux autres nations. Et puis, à quoi bon parler
de ce qui a disparu dans l'abîme des temps? L'histoire du genre humain
n'est que l'histoire de la bête humaine. Oublions les morsures que nous
nous sommes faites les uns les autres.

Durant leur guerre avec les Boërs de l'Afrique méridionale, les Anglais
durent à l'habileté et au courage de ces guerriers improvisés des revers
douloureux. Un de ceux qui leur firent le plus de mal est, comme on le
sait, le général Dewett. Or, un jour, le portrait de ce chef héroïque
parut tout à coup sur l'écran d'un cinématographe à Londres: la salle
entière applaudit à l'image du grand ennemi. Je me demande ce qui se
serait produit dans la même occurrence en quelque autre pays de
l'Europe. Oh, grand et noble peuple anglais, ne crains pas pour ton
immense empire! Les anges soutiennent de leurs ailes les puissances qui
sont justes!

Du contact intime de la France et de l'Angleterre, pays libres, la
Russie sortira imprégnée de l'esprit de liberté. Chose inouïe, on y
voit déjà un despote libérer son peuple. «Vous autres philosophes,
disait Catherine II à Diderot qui la poussait à faire des réformes, vous
écrivez sur du papier et le papier supporte très bien le frottement de
la plume; mais nous, les rois, c'est sur la peau humaine que nous
travaillons: elle est beaucoup plus sensible.» Le bon tsar Nicolas a une
belle occasion d'éprouver la sentence de son aïeule. Il existe dans son
vaste empire un parti réactionnaire qui crie comme nos «chisperos» du
siècle dernier: «Vivent les chaînes!» et qui a paralysé sa généreuse
initiative. En face de ce parti s'en dresse un autre, intransigeant,
féroce et qui prétend faire table rase de la tradition. Avec un diable
si déchaîné, il est bien difficile de sortir de l'enfer.

L'Italie gagnera Trieste. L'ombre de Sylvio Pellico, qui erre et gémit
encore à travers l'Italie irrédente, pourra se reposer en paix dans son
sépulcre. La Belgique aura bientôt étanché le sang de ses blessures. La
Turquie livrera aux chrétiens le tombeau du Christ. Les États
balkaniques continueront de se tirer par les cheveux en sourdine jusqu'à
ce que l'Europe, comme un maître rigoureux, portant un doigt à ses
lèvres et montrant sa férule, leur impose la paix.

Le désarmement s'ensuivra-t-il? Oui, j'espère que le désarmement
s'ensuivra. La maladie a produit une crise: le malade en mourra, ou il
sera sauvé. Redescendons dans les gouffres de l'animalité ou
élevons-nous au-dessus des nuages.

«L'animal prend son point d'appui sur la plante, dit M. Henri Bergson,
l'homme chevauche sur l'animalité, et l'humanité entière, dans l'espace
et dans le temps, est une immense armée qui galope à côté de chacun de
nous, en avant et en arrière de nous, dans une charge entraînante
capable de culbuter toutes les résistances et de franchir bien des
obstacles, même peut-être la mort.»

L'obstacle que l'humanité vient de rencontrer est le plus haut sur
lequel elle soit tombée dans sa longue carrière. Le tremplin est devant.
Si elle recule, nous continuerons de chevaucher, non pas devant, mais à
côté même de l'animal. Comme dans le fond de l'océan, c'est la loi du
plus fort qui continuera de s'appliquer. L'état de guerre se poursuivra
sur notre planète, la haine établira définitivement son empire sur les
cœurs; la bête rugira de nouveau par la bouche des canons. Si
l'humanité saute, elle tombera sur le doux sein de la loi du Christ,
elle acquerra pour toujours conscience de soi-même et poursuivra
glorieusement son chemin vers les hauts destins que la Providence lui a
réservés.

\end{chapter}